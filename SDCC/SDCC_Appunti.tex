\documentclass[a4paper,12pt]{report}
\usepackage[utf8]{inputenc}
\usepackage[italian]{babel}
\usepackage{hyperref}
\usepackage{amsmath}

% Titolo e Autore
\title{Sistemi Distribuiti e Cloud Computing}
\author{Valeria Cardellini}
\date{Anno Accademico 2024/2025}

\begin{document}

% Pagina del titolo
\maketitle
\tableofcontents

% Capitolo 1 - Introduzione ai sistemi distribuiti
\chapter{Introduzione ai Sistemi Distribuiti}
\label{chap:introduzione}

La \textbf{Legge di Metcalfe} afferma che il valore di una rete di telecomunicazione è proporzionale al quadrato del numero di utenti connessi al sistema. Per esempio, il potere economico di aziende come Facebook, Google e Instagram è cresciuto grazie a questa legge.

Negli ultimi anni, l'evoluzione tecnologica ha portato a computer:
\begin{itemize}
    \item Più piccoli
    \item Più economici
    \item Più efficienti dal punto di vista energetico
    \item Più veloci
\end{itemize}

Esempi di \textbf{sistemi distribuiti} includono:
\begin{itemize}
    \item Internet e Web
    \item Sistemi Cloud
    \item Sistemi peer-to-peer
    \item Internet of Things (IoT)
\end{itemize}

Secondo \textbf{Leslie Lamport}, un \textit{sistema distribuito} è un sistema in cui il fallimento di un computer, che non sapevamo esistesse, può rendere inutilizzabile il nostro sistema.

\section{Definizione di Sistema Distribuito}
\label{sec:definizione-sistema-distribuito}

Un \textit{sistema distribuito} è caratterizzato da:
\begin{enumerate}
    \item Un insieme di elementi computazionali autonomi, che appaiono esternamente come un unico sistema coerente grazie a un \textit{middleware}. Gli elementi autonomi, spesso chiamati \textit{nodi}, possono essere hardware o software.
    \item Comunicazione e coordinamento tra i componenti tramite scambio di messaggi (sincrono o asincrono).
    \item Un sistema dove il fallimento di un nodo può rendere inutilizzabile il servizio senza che ne fossimo a conoscenza.
\end{enumerate}

\textit{Leslie Lamport}, vincitore del Turing Award nel 2013, ha dato contributi fondamentali alla teoria e alla pratica dei sistemi distribuiti, introducendo concetti chiave come la causalità, i clock logici e i fallimenti bizantini [9:source].

\section{Utilità di un Sistema Distribuito}
\label{sec:utilita-sistema-distribuito}

I sistemi distribuiti sono utili per:
\begin{itemize}
    \item Condividere risorse (nodi computazionali, spazio di storage, rete, applicazioni).
    \item Migliorare la qualità del servizio, in termini di:
    \begin{itemize}
        \item Prestazioni (riduzione del tempo di risposta)
        \item Disponibilità (percentuale del tempo in cui il sistema è operativo)
    \end{itemize}
    \item Migliorare la sicurezza, grazie alla distribuzione dei nodi che riduce i rischi di un attacco centralizzato.
    \item Distribuire componenti del sistema su larga scala (anche geograficamente).
    \item Mantenere l'autonomia, evitando componenti centralizzate che possono rappresentare un punto di fallimento [9:source].
\end{itemize}

\section{Differenze tra Sistema Distribuito e Centralizzato}
\label{sec:differenze-sistema-distribuito}

Le principali differenze includono:
\begin{itemize}
    \item \textbf{Concorrenza}: Nei sistemi distribuiti, la concorrenza è inevitabile poiché i nodi lavorano in parallelo, mentre nei sistemi centralizzati può essere una scelta di design.
    \item \textbf{Mancanza di un Clock Globale}: In un sistema distribuito, ci sono molti clock fisici che non sono necessariamente sincronizzati tra loro.
    \item \textbf{Indipendenza}: Il fallimento di un thread in un sistema centralizzato può compromettere l'intero sistema, mentre nei sistemi distribuiti, il fallimento di un nodo non compromette necessariamente l'intero sistema.
\end{itemize}

\section{Caratteristiche di un Sistema Distribuito}
\label{sec:caratteristiche-sistema-distribuito}

Le principali caratteristiche di un sistema distribuito sono:
\begin{itemize}
    \item \textbf{Eterogeneità}: I nodi possono essere costituiti da differenti reti, sistemi operativi e linguaggi di programmazione. Il \textit{middleware} maschera questa eterogeneità.
    \item \textbf{Trasparenza}: L'utente non dovrebbe percepire la distribuzione delle risorse. I tipi di trasparenza includono:
    \begin{itemize}
        \item Accesso: Nasconde le differenze nella rappresentazione dei dati e nel modo in cui le risorse sono accedute.
        \item Locazione: Nasconde dove la risorsa è localizzata, per esempio un URL nasconde l'indirizzo IP [9:source].
        \item Migrazione e replicazione: Nasconde che le risorse possano essere spostate o replicate senza impattare l'operatività.
    \end{itemize}
    \item \textbf{Scalabilità}: La capacità di mantenere le prestazioni adeguate nonostante la crescita del numero di utenti o processi.
    \item \textbf{Apertura}: I sistemi distribuiti devono essere aperti, in grado di interagire con altri sistemi e garantire portabilità.
\end{itemize}

% Capitolo 2 - Virtualizzazione e Architettura
\chapter{Virtualizzazione e Architettura}
\label{chap:virtualizzazione}

\section{Pro e Contro}
\label{sec:pro-contro}

La virtualizzazione offre numerosi vantaggi, tra cui una maggiore flessibilità nell'utilizzo delle risorse e una riduzione dei costi operativi. Tuttavia, presenta anche alcune sfide, come la gestione della sicurezza e la complessità di configurazione.

\section{Architettura Xen}
\label{sec:architettura-xen}

La piattaforma \textit{Xen} è un esempio di hypervisor che permette la creazione di macchine virtuali con un controllo fine sull'allocazione delle risorse hardware.

% Capitolo 3 - Migrazione e Resizing
\chapter{Migrazione e Resizing}
\label{chap:migrazione-resizing}

\section{Resizing}
\label{sec:resizing}

Il \textit{resizing} è il processo di adattamento dinamico delle risorse assegnate a una macchina virtuale.

\section{Migrazione}
\label{sec:migrazione}

La migrazione delle macchine virtuali consente di spostare le istanze tra server fisici per bilanciare il carico o per manutenzioni programmate.

% Capitolo 4 - Lightweight Operating Systems e Unikernel
\chapter{Lightweight Operating Systems e Unikernel}
\label{chap:lightweight-os}

\section{Docker}
\label{sec:docker}

\textit{Docker} è una piattaforma di virtualizzazione a livello di sistema operativo che utilizza container per isolare le applicazioni in ambienti indipendenti.

\section{Kubernetes}
\label{sec:kubernetes}

\textit{Kubernetes} è una piattaforma di orchestrazione per container che automatizza il deployment, la gestione e la scalabilità delle applicazioni containerizzate.

% Capitolo 5 - Tecniche di Scaling nei Sistemi Distribuiti
\chapter{Tecniche di Scaling nei Sistemi Distribuiti}
\label{chap:scaling}

\section{Scaling Verticale e Orizzontale}
\label{sec:scaling-verticale-orizzontale}

Ci sono due principali direzioni di scaling nei sistemi distribuiti:
\begin{itemize}
    \item \textbf{Scaling verticale (scale-up)}: Aumentare la potenza delle risorse già esistenti.
    \item \textbf{Scaling orizzontale (scale-out)}: Aggiungere più risorse della stessa capacità [9:source].
\end{itemize}

\section{Tecniche di Scalabilità}
\label{sec:tecniche-scalabilita}

Per garantire la scalabilità, si possono utilizzare diverse tecniche:
\begin{itemize}
    \item Nascondere la latenza di comunicazione con la comunicazione asincrona.
    \item Spostare le computazioni il più vicino possibile ai client.
    \item Partizionare i dati e le computazioni, distribuendoli su più risorse (divide et impera).
    \item Replicare risorse e dati del sistema, rendendoli disponibili su diversi nodi [9:source].
\end{itemize}

\end{document}
