\documentclass[11pt,a4paper]{article}
\usepackage[utf8]{inputenc}
\usepackage[T1]{fontenc}
\usepackage[italian]{babel}
\usepackage{microtype}
\usepackage{amsmath,amssymb}
\usepackage{geometry}
\usepackage{hyperref}
\usepackage{parskip}
\usepackage{lmodern}
\usepackage{listings}
\usepackage{xcolor}
\usepackage{graphicx}
\usepackage{float}
\usepackage{booktabs}
\usepackage{tabularx}
\usepackage{longtable}
\usepackage{makecell}
\usepackage{caption}
\usepackage{array}
\captionsetup[table]{labelfont=bf,font=small}


\geometry{margin=2.5cm}

\lstset{
  basicstyle=\ttfamily\small,
  breaklines=true,
  frame=single,
  backgroundcolor=\color{gray!7},
  columns=fullflexible
}

\title{Fondamenti dell'Architettura Internet e Vulnerabilità Intrinseche dei Protocolli IP/TCP\\
\large (Appunti elaborati a partire dalle slide del corso \\ \normalsize Network and System Defense, A.A. 2025/2026)}
\author{Leonardo Polidori, Edoardo Marchionni, Chat GPT}

% Profondità dell’indice e della numerazione
\setcounter{tocdepth}{4}   % fino a \paragraph
\setcounter{secnumdepth}{20} % numera fino a \subparagraph

\begin{document}
\maketitle

\tableofcontents
\clearpage

\section{NET\_01}
\subsection{Architettura di base e principi di rete}

\subsubsection{Definizione e interconnessione}
Internet è un'\emph{inter-rete}: un insieme di numerose sotto-reti eterogenee connesse tra loro. Le sotto-reti possono essere basate su tecnologie diverse al Livello 2 (L2) — ad esempio 802.11 (WiFi), 802.3 (Ethernet), tecnologie cellulari (3G/4G), fibra ottica o ADSL — ma comunicano grazie a uno stack di protocolli comune, implementato sopra i diversi livelli fisici e MAC, nella logica del paradigma OSI. Il protocollo di base che abilita l'interoperabilità è l'Internet Protocol (IP), affiancato da protocolli di livello superiore come TCP, UDP e protocolli applicativi (per es. DNS, HTTP).

\subsubsection{Indirizzamento e instradamento}
Ogni dispositivo in una rete IP possiede un identificatore numerico univoco: l'indirizzo IP (32 bit per IPv4, 128 bit per IPv6). Le sotto-reti sono connesse mediante router, dispositivi che effettuano l'inoltro (forwarding) dei pacchetti dalla sorgente alla destinazione seguendo regole presenti nelle tabelle di instradamento.

Il forwarding si basa sul principio del \emph{Longest Prefix Match} (LPM): per decidere quale voce della routing table utilizzare si seleziona la corrispondenza con il prefisso di rete più lungo che include l'indirizzo di destinazione. L'inoltro può essere:
\begin{itemize}
  \item \emph{diretto}, quando la destinazione si trova nella stessa rete locale;
  \item \emph{indiretto}, quando la destinazione è raggiungibile tramite un next hop (salto successivo).
\end{itemize}

\begin{figure}[H]
  \centering
  \includegraphics[width=.85\linewidth]{immagini/slide1/stessa_rete.png}
  \caption{Instradamento diretto: sorgente e destinazione nella stessa rete locale (L2/L3).}
  \label{fig:stessa-rete}
\end{figure}

Il compito di IP è consegnare il pacchetto alla rete finale; la consegna al dispositivo specifico è rimandata al livello L2, che si occupa della mappatura IP <-> L2 (per esempio tramite ARP per IP <-> MAC).

\subsubsection{Anatomia dell'indirizzo IP e subnetting}
Le reti IP sono suddivise in subnet logiche. Due host appartenenti alla stessa subnet condividono i primi $X$ bit dell'indirizzo (la \emph{parte di rete}), mentre i restanti 32 $X$ bit identificano l'host. Dal 1984 si usa CIDR (Classless Inter Domain Routing): il prefisso non è più implicito per classi fisse ma viene specificato tramite una \emph{subnet mask} (o notazione ``/length''), che indica quali bit rappresentano la parte di rete. L'\textbf{i}-esimo bit della subnet mask è settato a \textbf{0} se i'\textbf{i}-esimo bit è nella host part; \textbf{1} se invece è nel prefisso network .Ad esempio, con indirizzo \texttt{192.168.1.12} e maschera \texttt{255.255.255.0} (ovvero \texttt{/24}) la rete è \texttt{192.168.1.0}.

\begin{figure}[H]
  \centering
  \includegraphics[width=.85\linewidth]{immagini/slide1/esempio_mask.png}
  \caption{Esempio network prefix}
  \label{fig:esempio_mask}
\end{figure}


Ogni rete ha due indirizzi ip \textbf{RISERVATI} ovvero:
\begin{itemize}
    \item\emph{Net address} tutti i bits nella parte host sono 0.
    \item\emph{Broadcast address} tutti i bits nella parte host sono 1.
\end{itemize}

\begin{figure}[H]
  \centering
  \includegraphics[width=.85\linewidth]{immagini/slide1/esempio_ip.png}
  \caption{trovare il network e broadcast address}
  \label{fig:esempio_ip}
\end{figure}

\subsubsection*{Esempio spiegato: trovare rete e broadcast di \texttt{209.85.129.99/27}}
 
Vogliamo ricavare l'indirizzo di \textbf{rete} e di \textbf{broadcast} per l'host \texttt{209.85.129.99/27}.

\paragraph{1) Cosa significa ``/27''}
La notazione \texttt{/27} dice che i \textbf{primi 27 bit} dell'indirizzo sono di \emph{rete} e i rimanenti sono di \emph{host}.
Nel formato ``a ottetti'', i primi 24 bit coincidono con i primi tre ottetti; quindi la /27 ``entra'' nel \textbf{quarto ottetto}:
\[
\underbrace{\text{[8] [8] [8]}}_{\text{rete}} \; \underbrace{\text{[3]}}_{\text{rete}} \; \underbrace{\text{[5]}}_{\text{host}}
\]
La maschera corrispondente è \texttt{255.255.255.224}, perché nell'ultimo ottetto i 3 bit di rete valgono \texttt{11100000\textsubscript{2}} $= 224$.

\paragraph{2) Perché basta guardare l'ultimo ottetto}
Con /27 i primi tre ottetti (\texttt{209.85.129}) restano identici per rete/host/broadcast.
Tutta la partizione in sottoreti avviene nel \textbf{quarto ottetto}. Qui rimangono \textbf{5 bit di host} $\Rightarrow$ ogni sottorete ha $2^5=32$ indirizzi contigui (un ``blocco'').

\paragraph{3) Trova il blocco in cui cade 99}
I blocchi nel quarto ottetto sono a passi di 32: \texttt{0--31}, \texttt{32--63}, \texttt{64--95}, \texttt{96--127}, \texttt{128--159}, \dots
Poiché \texttt{99} appartiene a \texttt{96--127}, la nostra \textbf{rete} è \texttt{209.85.129.\underline{96}} e il \textbf{broadcast} sarà l'ultimo del blocco, \texttt{209.85.129.\underline{127}}. Verifichiamo con l'AND bit-a-bit.

\paragraph{4) Verifica con l'AND (ultimo ottetto)}.

\begin{lstlisting}
[basicstyle=\ttfamily\small,frame=single]
99   = 01100011
224  = 11100000   (maschera /27 nell'ultimo ottetto)
AND    --------
       01100000 = 96   --> indirizzo di rete (quarto ottetto)
\end{lstlisting}
Quindi l'indirizzo di \textbf{rete} è \texttt{209.85.129.96}.

\paragraph{5) Broadcast: tutti i bit host a 1}
Nel broadcast si \emph{mantengono} i 3 bit di rete e si mettono a 1 i 5 bit di host:
\[
\text{rete (ult.\ ottetto)} = 01100000 \quad + \quad 00011111 \; (\text{tutti i bit host a 1}) \;=\; 01111111 = 127.
\]
Dunque \textbf{broadcast} = \texttt{209.85.129.127}.

\paragraph{6) Intervallo host e conteggio}
Gli host validi sono i numeri compresi \emph{tra} rete e broadcast:
\[
\texttt{209.85.129.97} \; \text{fino a} \; \texttt{209.85.129.126},
\]
per un totale di $2^5 - 2 = 30$ host (si escludono rete e broadcast).






\subsubsection{Sistemi autonomi (AS) e routing globale}
L'inter-rete è organizzata in Autonomous Systems (AS), domini amministrativi che gestiscono internamente le proprie politiche di instradamento. All'interno di un AS si adottano Interior Gateway Protocols (IGP) come OSPF, IS-IS o RIP; lo scambio di rotte tra AS diversi avviene tramite l'Exterior Gateway Protocol più usato: BGP, che garantisce la raggiungibilità globale.

\subsubsection{La routing table e l'algoritmo di lookup}
La routing table contiene entry costituite tipicamente da: indirizzo di destinazione, maschera/netmask, next hop e interfaccia d'uscita. La funzione di lookup per un pacchetto $p$ itera le voci ordinate per lunghezza del prefisso e restituisce la voce $i$ tale che
\[
(p.daddr \;\&\; i.mask) = i.addr,
\]
dove \texttt{\&} è l'AND bit-a-bit; se non si trova alcuna corrispondenza il pacchetto viene scartato.

\subsection{II. Il viaggio del pacchetto: esempio di richiesta DNS}

\subsubsection{Topologia e hop}
Un pacchetto generato da un browser per risolvere un nome (ad esempio \texttt{www.google.com}) percorre una serie di hop: rete domestica (WiFi), access point/router, edge router dell'AS dell'utente, router di confine (border router), una sequenza di AS di transito e infine il data center che ospita il servizio DNS o il server web. Ogni tratto può utilizzare tecnologie e politiche diverse, e rappresenta un potenziale punto di vulnerabilità.

\begin{figure}[H]
  \centering
  \includegraphics[width=.85\linewidth]{immagini/slide1/dns.png}
  \caption{dal web browser al web server}
  \label{fig:dns}
\end{figure}

\subsubsection{Stack protocollare e incapsulamento}
La richiesta DNS attraversa gli strati della pila:

\begin{itemize}
  \item \textbf{Livello applicativo (DNS)}: genera la query di risoluzione ("dammi l'indirizzo ip di www.google.com").
  \item \textbf{Livello trasporto (UDP)}: aggiunge porta sorgente (es. 5000) e porta destinazione (53), oltre al checksum.
  \item \textbf{Livello rete (IP)}: inserisce indirizzi IP sorgente e destinazione (es. \texttt{10.0.0.100} e \texttt{85.18.200.200}), TTL, eventuale fragmentation.
  \item \textbf{Livello accesso (L2)}: incapsula il frame con indirizzi MAC del next hop; questi cambiano ad ogni salto.
\end{itemize}

Durante il percorso gli indirizzi IP rimangono costanti, mentre gli indirizzi MAC vengono aggiornati hop-by-hop. Meccanismi come ARP permettono la traduzione dinamica IP↔MAC all'interno di una stessa rete locale.



\begin{figure}[H]
  \centering
  \includegraphics[width=.85\linewidth]{immagini/slide1/dns2.png}
  \caption{richiesta DNS (semplificata)}
  \label{fig:dns2}
\end{figure}



\subsection{III. Vulnerabilità intrinseche di IP e TCP}

I protocolli storici di Internet sono stati progettati principalmente per interoperabilità e scalabilità, non per la sicurezza. Questo ha lasciato diverse debolezze intrinseche.

\subsubsection{Identificazione, spoofing e non-ripudio}
Gli \textbf{identificatori di rete} (indirizzi IP e MAC) sono semplici stringhe binarie facilmente manipolabili: un mittente può generare pacchetti con sorgente falsata (\textbf{IP spoofing}) oppure modificare l'indirizzo sorgente di pacchetti che sta inoltrando. Questo fenomeno rende possibile, ad esempio, l'impersonificazione di server legittimi (un attaccante può inviare pacchetti che sembrano provenire da un DNS server affidabile). \textbf{IP non fornisce meccanismi di autenticazione dell'origine}: non esiste un modo intrinseco per dimostrare che l'indirizzo sorgente di un pacchetto corrisponda realmente al mittente fisico, provocando problemi di \emph{repudiation}.

\subsubsection{Confidenzialità}
\textbf{Il protocollo IP non cifra il payload né fornisce protezione contro l'intercettazione}: catturare e decodificare pacchetti su un segmento di rete è, in molti casi, semplice. Inoltre gli utenti non controllano l'intero percorso seguito dai pacchetti; anche fidandosi del proprio ISP, sono necessari fiducia e verifiche su tutti gli AS attraversati. Attacchi di route hijacking o route leaking possono alterare il percorso e compromettere la riservatezza.

\subsubsection{Integrità dei dati}
IP, TCP e UDP usano checksum per rilevare errori di trasmissione (header e payload), ma questi meccanismi non sono progettati come primitive di sicurezza: sono vulnerabili a manipolazioni intenzionali poiché basta ricalcolare il checksum dopo la modifica del pacchetto. Per esempio, il checksum IP è una semplice somma/XOR su parole dell'header e non offre garanzie contro un attaccante attivo.



\subsubsection{Packet replication e anti-replay}
A livello IP \textbf{non} esistono numeri di sequenza o marcatori univoci che identifichino inequivocabilmente un pacchetto in un flusso; il problema anti-replay è quindi in gran parte non risolto a questo livello. TCP fornisce numeri di sequenza, ma essi sono destinati alla gestione dell'affidabilità e dell'ordine, non all'autenticazione. Poiché tali numeri non sono protetti criptograficamente, possono essere predetti o spoofati in alcuni scenari, consentendo replay o session hijacking se non vengono adottate contromisure a livello superiore (ad esempio TLS).

\subsubsection{Insicurezza delle mappature dinamiche}
Molti servizi critici si basano su mappature dinamiche non progettate per la sicurezza: DNS (nomi→IP), ARP (IP→MAC), tabelle di bridging (MAC→porta), e la stessa routing table (destinazione→next hop). Implementazioni legacy, come il DNS non autenticato, permettono a un attaccante di fornire risposte fasulle: la risoluzione nome→IP non è intrinsecamente verificabile senza meccanismi come DNSSEC.

\newpage
\maketitle
{\LARGE \textbf{Laboratorio}} \\[0.5em]
\subsection{Laboratorio 1: MiTM e DNS spoofing}

\subsubsection{Obiettivo e scenario}
L'obiettivo è dirottare richieste HTTP non cifrate attraverso DNS spoofing e impersonificazione di un sito (es. \texttt{http://netgroup.uniroma2.it}). Lo scenario tipico prevede un attaccante nella stessa rete locale della vittima; il resolver della vittima è configurato su un DNS pubblico (per es. \texttt{8.8.8.8}). L'attacco combina:
\begin{enumerate}
  \item lo sfruttamento della mappatura IP <-> MAC (via ARP spoofing) per stabilire un MiTM;
  \item la manipolazione delle risposte DNS per risolvere il nome del sito bersaglio verso un indirizzo controllato dall'attaccante.
\end{enumerate}


\begin{figure}[H]
    \centering
    \includegraphics[width=.85\linewidth]{immagini/slide1/lab1.png}
    \caption{Topologia rete}
    \label{fig:lab1}
\end{figure}

\subsubsection{Fasi dell'attacco}
L'attacco tipico è articolato in quattro step principali:
\begin{enumerate}
  \item \textbf{STEP 1 – MiTM} (ARP spoofing): l'attaccante dissipa nelle cache ARP della vittima e del gateway risposte ARP falsificate in modo da farsi passare per entrambi e intercettare il traffico.
  \item \textbf{STEP 2 – Intercettazione della richiesta DNS}: una volta in posizione di MiTM, l'attaccante intercetta le query DNS emesse dalla vittima.
  \item \textbf{STEP 3 – Spoofing della risposta DNS}: l'attaccante risponde con una risoluzione fasulla per il dominio bersaglio, puntando a un IP di sua proprietà.
  \item \textbf{STEP 4 – Impersonificazione del sito web}: l'attaccante serve una copia del sito (ottenuta tramite mirroring) dall'IP di controllo, così che la vittima riceva contenuti apparentemente legittimi.
\end{enumerate}

\subsubsection{STEP 1: ARP spoofing (MiTM)}
L'attaccante invia risposte ARP non richieste (opcode 2) sia alla vittima che al default gateway:
\begin{itemize}
  \item alla vittima: un frame ARP unicast indirizzato al MAC della vittima (\texttt{bb:bb:bb:bb:bb:bb}) affermando che l'IP del router (\texttt{10.0.0.1}) corrisponde al MAC dell'attaccante (\texttt{aa:aa:aa:aa:aa:aa});
  \item al gateway: un frame ARP unicast indirizzato al MAC del router (\texttt{cc:cc:cc:cc:cc:cc}) affermando che l'IP della vittima (\texttt{10.0.0.100}) corrisponde al MAC dell'attaccante (\texttt{aa:aa:aa:aa:aa:aa}).
\end{itemize}
Ripetendo queste risposte periodicamente, l'attaccante mantiene la posizione di MiTM.

\subsubsection{STEP 2 \& 3: intercettazione e DNS spoofing}
Dopo aver stabilito il MiTM, l'attaccante può reindirizzare le richieste DNS verso la sua macchina:

\begin{lstlisting}[language=bash, caption={Esempio: regola iptables per reindirizzare richieste DNS (UDP 53) alla macchina locale}]
iptables -t nat -A PREROUTING -p udp --dport 53 -j REDIRECT
\end{lstlisting}

Sulla macchina dell'attaccante viene eseguito un server DNS leggero (es. \texttt{dnsmasq}) con una configurazione del tipo:

\begin{lstlisting}[caption={Estratto di /etc/dnsmasq.conf}]
interface=eth0
no-dhcp-interface=eth0
server=1.1.1.1

# Risolvi il dominio bersaglio verso l'IP dell'attaccante
address=/netgroup.uniroma2.it/10.0.0.200
\end{lstlisting}

Questa configurazione restituisce per \texttt{netgroup.uniroma2.it} l'indirizzo \texttt{10.0.0.200}; tutte le altre query vengono inoltrate al resolver pubblico (qui \texttt{1.1.1.1}).

\subsubsection{STEP 4: impersonificazione del sito web}
L'attaccante può aver replicato il contenuto del sito bersaglio tramite strumenti di mirroring, ad esempio:

\begin{lstlisting}[language=bash]
wget --mirror --convert-links --html-extension --no-parent -l 1 \
     --no-check-certificate http://netgroup.uniroma2.it
\end{lstlisting}

I contenuti mirrorati vengono serviti localmente (per es. con Apache2). Così la vittima, ricevendo l'IP dell'attaccante per il dominio richiesto, ottiene una copia apparentemente autentica del sito.

\subsection*{Conclusione e contromisure (sintesi)}
Le vulnerabilità descritte evidenziano che senza meccanismi di autenticazione, integrità e confidenzialità a livello superiore, l'infrastruttura IP/TCP è esposta a compromissioni. Contromisure pratiche includono:
\begin{itemize}
  \item utilizzo diffuso di canali cifrati e autenticati (TLS/HTTPS) per proteggere le applicazioni;
  \item adozione di estensioni e protocolli progettati per la sicurezza (es. DNSSEC per autenticare risposte DNS, IPsec per integrità/confidenzialità a livello IP dove applicabile);
  \item tecniche di difesa a livello di rete locale (ARP inspection, dynamic ARP protection, filtraggio di pacchetti spoofati sui router e access control lists);
  \item pratiche operative: aggiornamento dei software, monitoraggio delle anomalie di routing e validazione delle rotte BGP.
\end{itemize}




%%%%%%%%%%%%%%%%%% NET_02 %%%%%%%%%%%%%%%%%%%%%
\newpage
\maketitle
\section{NET\_02}

\subsection{Sicurezza delle Reti Ethernet LAN (Livello 2)}

\subsection{Perché la LAN Ethernet è fragile per natura}
Ethernet nasce per \emph{autoconfigurarsi}: gli switch imparano indirizzi e percorsi (MAC learning), i protocolli di controllo (STP, ARP, DHCP) si basano su broadcast e sulla mancanza di autenticazione a L2. Questo rende immediati tre vettori: \textbf{osservare} (eavesdropping), \textbf{manipolare} (spoofing/MITM) e \textbf{interrompere} (DoS). Le minacce si organizzano in quattro famiglie: (i) accesso a rete/sistemi, (ii) confidenzialità, (iii) disponibilità, (iv) integrità.

\subsubsection{Frame, indirizzamento e forwarding}
Gli indirizzi Mac sono a 48 bit e ognuno di essi ha un utilizzo specifico.\\
nella versione originale dell ethernet la tipologia del frame veniva usata per il demultiplexing del layer superiore per esemio 0x800=IP; mentre in 802.3 indica la lunghezza oppure il tipo in particolare se il frame supera i $0x0600$ ($1536_{10}$) allora indica la tipologia di frame altrimenti LLC per il demultiplexing e indica il payload size. Se frame inferiore ai 46 bit allora viene utilizzato del padding.\\
Il primo bit dell’indirizzo MAC indica se l’indirizzo è unicast (0) o di gruppo/multicast (1); il secondo bit distingue tra indirizzi globali (0, assegnati dal produttore) e locali (1, configurabili via software o driver).

\paragraph{Multiport Repeaters (Hub)}
Gli hub, o ripetitori multiporta, operano come un bus condiviso: tutto il traffico ricevuto viene rigenerato su tutte le porte. 
Appartengono quindi a un unico \textbf{dominio di collisione} e non offrono isolamento tra host; per questo sono oggi sostituiti dagli switch.\\

\paragraph{Bridge/Switches}
Gli switch possono operare in:
\begin{itemize}
  \item \textbf{Store\&Forward}: lettura completa del frame (memorizzazione su buffer), controllo CRC, scarto dei frame \emph{runt}(<64 bytes too short)/troppo lunghi oppure se fallisce il CRC; look up nella tabella e forwarding.
  \item \textbf{Cut-through}: lettura solo fino all'indirizzo, nessun check di integrità, look-up, forwarding.
\end{itemize}
Il \textbf{Forwarding Database} (FDB) mappa \texttt{MAC→porta}; le entry dinamiche sono apprese (MAC learning) e scadono con \emph{ageing} tipicamente nell’ordine dei 300\,s; altrimenti impostate staticamente da un sysadmin o da un db. Se la destinazione è sconosciuta, lo switch effettua \emph{flooding}.
\paragraph{Address Learning}
Un frame arriva alla porta X quindi deve provenire dalla LAN connessa dalla porta X, il source address viene usato per l update del forwarding DB. Se arriva un frame da un source addr non presente nella tabella allora viene creata la entry con $age=0$; se invece arriva un entry già presente viene refreshata la age di quella entry.

\begin{figure}[H]
    \centering
    \includegraphics[width=.85\linewidth]{immagini/NET_02/adress_learning_1.png}
    \caption{Topologia rete}
    \label{fig:lab1}
\end{figure}
Infine se arriva un frame da un source addr già presente nella tabella ma sotto una porta differente allora viene aggiornata la entry e refresh della age.

\begin{figure}[H]
    \centering
    \includegraphics[width=.85\linewidth]{immagini/NET_02/address_learning_2.png}
    \caption{Topologia rete}
    \label{fig:lab1}
\end{figure}


\subsubsection{Topologie e controllo dei loop: STP}
Quando in una rete Ethernet esistono collegamenti ridondanti, possono formarsi \emph{loop} a \textbf{Livello 2 (L2, Data Link)}: i frame possono circolare all’infinito tra switch, saturando la rete. 
Per evitare questo, entra in gioco lo \textbf{Spanning Tree Protocol}. 
L’idea è eleggere uno \emph{switch radice} (\textbf{root bridge}) e disattivare alcune porte in modo che la topologia effettiva sia un albero evitando così loop, pur lasciando i link ridondanti pronti a subentrare in caso di guasti.

Gli switch si mettono d’accordo scambiandosi messaggi di controllo chiamati \textbf{BPDU (Bridge Protocol Data Unit)}. 
Ogni BPDU contiene l’identità dello switch (\emph{Bridge ID}, che include \emph{priorità} e MAC) e i \emph{costi} dei percorsi. 
Viene eletto un \emph{root bridge}; e poi per ogni altro switch, \textbf{STP} sceglie:
\begin{itemize}
  \item una \textbf{porta radice (root port)}: la porta verso il root bridge con costo minore;
  \item eventuali porte in eccesso vengono messe in stato \textbf{bloccato (blocking)} per rompere i cicli.
\end{itemize}
Il risultato è che solo alcune porte sono di forwarding, mentre le altre restano \textbf{bloccate}; se un link o uno switch si guastano allora STP riconfigura la topologia evitando nuovamente loop.

Sul piano della sicurezza, però, STP ha un limite: a L2 non c’è \textbf{autenticazione} dei messaggi di controllo. 
Un host malevolo può inviare \textbf{BPDU} artefatte e farsi eleggere \emph{root bridge} (alzando la priorità o manipolando i costi), dirottando o interrompendo il traffico. 
Per questo, in produzione si usano contromisure come \emph{BPDU Guard}, \emph{Root Guard}, \emph{portfast} solo sugli host, e piani di controllo isolati.


\subsubsection{Adattamento del Livello 3 su Livello 2: DHCP, ARP e NDP}

L’interoperabilità tra il \textbf{Livello 2 (Data Link)} e il \textbf{Livello 3 (Network)} è assicurata da una serie di protocolli di adattamento che consentono agli host di configurare automaticamente i propri parametri IP e di risolvere gli indirizzi fisici (MAC) dei dispositivi vicini. 
Questi protocolli, fondamentali per il funzionamento delle reti IP, nascono però in un contesto di fiducia implicita e \textbf{assenza di autenticazione}, diventando quindi bersagli ideali per attacchi di spoofing e manipolazione del traffico.

\paragraph{DHCP — Dynamic Host Configuration Protocol (IPv4).}
Il \textbf{DHCP} automatizza l’assegnazione degli indirizzi IP e dei parametri di rete,. 
Opera in modalità \emph{client–server} e sfrutta il meccanismo di broadcasting a livello Ethernet per raggiungere il server anche quando il client non ha ancora un IP. 

Il protocollo segue un tipico \textbf{handshake a quattro fasi}:
\begin{enumerate}
  \item \textbf{DHCP Discover} — il client trasmette in broadcast (\texttt{255.255.255.255}) per cercare un server disponibile;
  \item \textbf{DHCP Offer} — il server risponde offrendo un indirizzo IP e altri parametri (gateway, DNS, lease time);
  \item \textbf{DHCP Request} — il client accetta esplicitamente un’offerta specifica;
  \item \textbf{DHCP ACK} — il server conferma l’assegnazione e crea una voce di \emph{lease} nel proprio database.
\end{enumerate}
Ogni lease è temporaneo e può essere rinnovato tramite messaggi \emph{DHCP Renew/Rebind}.  
Quando un host si trova in una rete diversa dal server, la comunicazione avviene tramite un \textbf{DHCP relay agent} — spesso un router — che incapsula i messaggi Discover/Request e li inoltra verso il server remoto, mantenendo così la visibilità dell’origine (campo \texttt{giaddr}).  
\begin{figure}
    \centering
    \includegraphics[width=0.5\linewidth]{immagini//NET_02/DHCP.png}
    \label{fig:placeholder}
\end{figure}

\textbf{Sicurezza:} DHCP non autentica né il client né il server. Un host malevolo può rispondere più velocemente del server legittimo (\textbf{DHCP spoofing}) e assegnare gateway o DNS controllati, dirottando il traffico.  
Per mitigare questi scenari, gli switch moderni implementano \textbf{DHCP Snooping}: una funzione che registra le associazioni IP–MAC–porta–VLAN apprese dai messaggi DHCP legittimi, utile anche per alimentare altri controlli di sicurezza come la \emph{Dynamic ARP Inspection}.

\paragraph{ARP — Address Resolution Protocol (IPv4).}
L'\textbf{ARP} consente di scoprire l’indirizzo MAC associato a un indirizzo IP all’interno della stessa LAN.  
Quando un host deve inviare un pacchetto IP verso una destinazione della propria subnet, interroga la rete inviando un messaggio \textbf{ARP Request} in broadcast contenente l’indirizzo IP cercato.  
L’host corrispondente risponde con un \textbf{ARP Reply} unicast, fornendo il proprio MAC address.  
Ogni sistema mantiene una \textbf{cache ARP} che memorizza temporaneamente queste associazioni per ridurre il numero di richieste future (tipicamente 20 minuti su sistemi UNIX-like).
\begin{figure}
    \centering
    \includegraphics[width=0.5\linewidth]{immagini//NET_02/ARP.png}
    \caption{}
    \label{fig:placeholder}
\end{figure}

\textbf{Debolezze:} ARP è \emph{stateless} e non prevede autenticazione.  
Un attaccante può quindi inviare \textbf{ARP Reply falsificati} (anche senza richiesta) per associare un IP legittimo al proprio MAC address: è il classico \textbf{ARP poisoning}, che consente di intercettare, modificare o bloccare il traffico (attacco \emph{Man-in-the-Middle}).  
Meccanismi come \textbf{Dynamic ARP Inspection (DAI)}, basati sulle tabelle di DHCP Snooping, permettono di bloccare risposte ARP incoerenti rispetto alle associazioni IP–MAC note.

\paragraph{NDP — Neighbor Discovery Protocol (IPv6).}
Nel mondo IPv6, il \textbf{Neighbor Discovery Protocol (NDP)}, sostituisce ARP e parte del ruolo di DHCP.  
Basato su \textbf{ICMPv6}
A differenza di ARP, NDP usa \textbf{multicast} anziché broadcast, riducendo l’impatto sulla rete e migliorando la scalabilità.  
Tuttavia, condivide la stessa assenza di autenticazione: un host può fingere di essere un router o un vicino legittimo, portando a \emph{neighbor spoofing} o \emph{router advertisement flooding}.
\subsubsection{Vulnerabilità e minacce principali}

La sicurezza di una rete Ethernet dipende fortemente dall’affidabilità del Livello 2, ma i protocolli su cui si basa — come ARP, DHCP e STP — sono stati progettati per un ambiente fidato, senza autenticazione o cifratura. 
Questo rende l’intera architettura LAN intrinsecamente vulnerabile ad attacchi che puntano al controllo dell’accesso fisico, alla manipolazione del traffico e al degrado delle prestazioni. 
Le principali minacce si raggruppano in quattro categorie: \textbf{(1) accesso alla rete}, \textbf{(2) riservatezza del traffico}, \textbf{(3) integrità e manipolazione}, \textbf{(4) disponibilità e prestazioni}.

\paragraph{1) Accesso alla rete.}

\textbf{Accesso fisico non autorizzato.}  
L’attacco più basilare consiste nel collegarsi fisicamente a una porta Ethernet lasciata attiva e non monitorata. 
In ambienti non presidiati (aule, uffici, open space), un attaccante può semplicemente inserire il proprio dispositivo o un piccolo switch/access point (\emph{rogue device}), espandendo la rete interna senza autorizzazione.  
Poiché lo standard Ethernet non prevede autenticazione a livello di porta, la connessione risulta immediatamente operativa.  
Questo tipo di “\emph{join} fisico” rappresenta il punto d’ingresso di molte compromissioni LAN.

\textbf{Accesso remoto e ricognizione.}  
Una volta ottenuto l’accesso (fisico o logico), l’attaccante può mappare la topologia di rete sfruttando protocolli di base:
\begin{itemize}
  \item \emph{ARP scanning} — per enumerare gli indirizzi IP attivi nella LAN;
  \item richieste \emph{DHCP} — per dedurre l’intervallo di indirizzi disponibili e i parametri di rete (gateway, DNS);
  \item \emph{port scanning} — per identificare i servizi esposti e i sistemi operativi in uso.
\end{itemize}
Queste informazioni consentono di costruire una mappa logica della rete e di selezionare successivi obiettivi di attacco (\emph{target profiling}).

\textbf{Compromissione e controllo dello switch.}  
Gli switch Ethernet possono essere presi di mira direttamente se le loro interfacce di gestione sono accessibili. 
Molti dispositivi vengono distribuiti con credenziali predefinite o con protezioni minime; un attaccante che riesce a ottenere l’accesso può:
\begin{itemize}
  \item modificare la configurazione VLAN;
  \item attivare porte di mirroring per intercettare traffico;
  \item manipolare parametri di Spanning Tree Protocol (STP) per diventare \emph{root bridge} e reindirizzare il traffico;
  \item disattivare link o generare \emph{Denial of Service}.
\end{itemize}
Inoltre, un reset fisico del dispositivo può riportarlo alle impostazioni di fabbrica, consentendo un takeover completo.

\paragraph{2) Riservatezza e intercettazione.}

\textbf{Eavesdropping (intercettazione).}  
In una rete Ethernet tradizionale basata su hub, tutto il traffico viene propagato su tutte le porte, rendendo triviale l’intercettazione passiva (\emph{sniffing}).  
Anche negli switch moderni, l’attacco resta possibile tramite:
\begin{itemize}
  \item installazione fisica di un dispositivo di ascolto (\emph{tap}) su un cavo in rame o fibra ottica;
  \item abilitazione di una scheda di rete in \emph{promiscuous mode} per ricevere frame non destinati al proprio MAC;
  \item abuso della funzione di \emph{port mirroring} sugli switch, utile per intercettare il traffico di altre porte.
\end{itemize}

\textbf{MAC flooding.}  
Gli switch memorizzano le associazioni \texttt{MAC→porta} nella CAM (Content Addressable Memory) o FDB (Forwarding Database).  
Un attaccante può saturare questa memoria inviando migliaia di frame con indirizzi MAC sorgente casuali.  
Quando la tabella si riempie, lo switch passa alla modalità \emph{flooding}, inoltrando i frame su tutte le porte come un hub.  
Questo permette di catturare traffico unicast normalmente privato e, in certi casi, di rompere l’isolamento tra VLAN (\emph{cross-VLAN leakage}).

\textbf{MAC spoofing.}  
Manipolando l’indirizzo MAC sorgente dei frame inviati, un attaccante può sostituirsi a un host legittimo già presente nella FDB.  
Il risultato è un \emph{hijacking} del traffico diretto alla vittima, che può essere intercettato, modificato o semplicemente bloccato.  
Lo spoofing è efficace perché Ethernet non verifica la coerenza tra il MAC dichiarato nel frame e quello della scheda di rete.

\paragraph{3) Integrità del traffico e attacchi Man-in-the-Middle (MITM).}

\textbf{ARP poisoning.}  
Approfittando del fatto che l’ARP accetta qualunque risposta non autenticata, un attaccante può inviare \emph{gratuitous ARP replies} falsificati per associare il proprio MAC address all’indirizzo IP di un altro nodo (ad esempio il gateway).  
In questo modo intercetta tutto il traffico tra la vittima e il router, realizzando un classico attacco \emph{Man-in-the-Middle (MITM)}.  
Varianti di questo attacco esistono anche in IPv6 sotto forma di \emph{Neighbor Advertisement spoofing} contro NDP.

\textbf{DHCP poisoning.}  
Simile nel principio, l’attacco DHCP poisoning consiste nel rispondere alle richieste DHCP più rapidamente del server legittimo.  
Il client riceve così parametri falsi (indirizzo IP, gateway, DNS), che permettono all’attaccante di deviare il traffico verso sistemi controllati o intercettarlo.

\textbf{Session hijacking.}  
Una volta intercettato il traffico (via ARP o DHCP poisoning), è possibile analizzare le sessioni di livello superiore (ad esempio TCP) e riprodurle, utilizzando numeri di sequenza o cookie d’autenticazione per impersonare un utente o un servizio.

\textbf{Replay attack.}  
Consiste nel riutilizzare pacchetti validi intercettati in precedenza — ad esempio messaggi di controllo o di autenticazione — per ottenere accesso o indurre comportamenti anomali nei dispositivi di rete.  
In mancanza di firme digitali o timestamp, questi messaggi vengono accettati come legittimi.

\paragraph{4) Disponibilità e attacchi di Denial of Service (DoS).}

\textbf{STP DoS e manipolazione topologica.}  
Il protocollo \textbf{Spanning Tree Protocol (STP, IEEE 802.1D)} non prevede autenticazione dei messaggi di controllo (\textbf{BPDU, Bridge Protocol Data Units}).  
Un attaccante può sfruttare questa debolezza per:
\begin{itemize}
  \item eleggersi come \emph{root bridge} forzando il traffico a passare attraverso il proprio nodo;
  \item inondare la rete di BPDU falsi, causando continui ricalcoli dell’albero di spanning e interruzioni periodiche dei collegamenti (\emph{STP reconvergence loops}).
\end{itemize}

\textbf{Resource exhaustion e flooding.}  
Un altro vettore comune consiste nel sovraccaricare il piano di controllo degli switch o dei router.  
Attacchi di \emph{unknown-unicast flooding} e tempeste di broadcast (\emph{broadcast storms}) possono saturare la banda o la memoria, impedendo il corretto forwarding dei frame.  
Simili risultati possono essere ottenuti generando una quantità eccessiva di richieste ARP o DHCP, portando al collasso del servizio (\emph{DoS a livello L2}).

\paragraph{Sintesi.}
In sintesi, le reti Ethernet soffrono di vulnerabilità strutturali dovute alla loro progettazione originaria: assenza di autenticazione, uso esteso di broadcast, apprendimento dinamico dei MAC e protocolli di controllo non cifrati.  
Molti di questi problemi sono oggi mitigabili grazie a funzioni avanzate degli switch — come DHCP Snooping, Dynamic ARP Inspection, BPDU Guard e Port Security — ma nessuna misura singola è sufficiente.  
La sicurezza del livello 2 è quindi il risultato di una \textbf{difesa stratificata}, che combina controllo d’accesso, segmentazione, monitoraggio continuo e buone pratiche di configurazione.

\subsubsection{Contromisure: dal minimo sindacale al robusto}

Per affrontare le vulnerabilità strutturali di Ethernet e le minacce che ne derivano, è fondamentale adottare un approccio stratificato. Le contromisure spaziano da soluzioni basilari, come la segmentazione della rete, a soluzioni avanzate che includono l'uso di crittografia L2 e l'autenticazione a livello di porta.

\paragraph{1) Segmentazione della rete.}

\textbf{Router-based security.}  
Una delle strategie di base per migliorare la sicurezza di una rete Ethernet è utilizzare un router per separare i vari domini L2. I router creano segmenti L3 separati, eliminando così il dominio di collisione condiviso tra tutti i dispositivi. Questo spezza il traffico di controllo come ARP, STP e VLAN, riducendo significativamente la superficie di attacco. Se il traffico è confinato all'interno di un segmento L2, un attacco che comprometta una porzione della rete non può propagarsi agli altri segmenti.

Un esempio pratico: se si utilizza un router per separare VLAN diverse, attacchi come l'iniezione di traffico non autorizzato o ARP poisoning sono limitati a un solo segmento, riducendo il rischio di compromissione intersegmento.

\textbf{VLAN perimetriche.}  
Le \textbf{VLAN} (Virtual Local Area Network) sono fondamentali per isolare il traffico tra dispositivi all'interno della stessa rete fisica. Utilizzare VLAN per limitare il \emph{blast radius} (raggio di propagazione di un attacco) è una delle prime linee di difesa contro le minacce interne. La configurazione corretta dei trunk, l'assegnazione adeguata delle VLAN perimetrali, e la disabilitazione dei protocolli di gestione su porte access sono passi essenziali.  
Ad esempio, mantenere VLAN 1 come VLAN nativa su tutte le interfacce trunk può rendere vulnerabile la rete agli attacchi di tipo VLAN hopping.

\paragraph{2) Controllo d’accesso.}

\textbf{802.1X (NAC) su porta.}  
L'uso di \textbf{802.1X} con un sistema di \textbf{Network Access Control (NAC)} fornisce un livello di sicurezza aggiuntivo, garantendo che solo dispositivi autenticati possano accedere alla rete. Con \textbf{RADIUS} come server di autenticazione, il dispositivo che tenta di connettersi viene autenticato tramite EAP (Extensible Authentication Protocol), permettendo l'accesso solo a dispositivi legittimi.  
Un esempio pratico: ogni dispositivo, al momento del collegamento, deve inviare le proprie credenziali via EAPOL (EAP over LAN). Il server RADIUS verifica la validità delle credenziali, abilitando l'accesso alla porta dello switch solo se autenticato.

**Limitazioni**: Sebbene 802.1X protegga contro il \emph{MAC spoofing} e il flooding, non è completamente immune a attacchi avanzati come l'\emph{ARP poisoning}, né impedisce un attacco \emph{piggybacking} dove un attaccante inserisce un mini-switch tra il dispositivo legittimo e lo switch autenticato.

\textbf{Port Security e ACL.}  
\textbf{Port Security} consente di limitare il numero di indirizzi MAC che possono essere appresi da una singola porta dello switch. Ciò impedisce attacchi come il \textbf{MAC flooding}, che saturano la memoria CAM dello switch. Inoltre, l'uso di \textbf{Access Control Lists (ACL)} per il filtraggio dei pacchetti a livello di switch, basato su indirizzi MAC o Ethertype, offre un ulteriore livello di protezione contro attacchi di tipo \emph{flooding} o \emph{spoofing}.  
Un esempio pratico: configurare una porta per consentire solo due dispositivi MAC, impedendo a un terzo dispositivo di connettersi.

```bash
# Esempio di ebtables per binding MAC→porta
ebtables -A FORWARD --in-interface eth0 -s ! a0:...:a0 -j DROP
ebtables -A FORWARD --in-interface eth1 -s ! b0:...:b0 -j DROP
ebtables -A FORWARD --in-interface eth2 -s c0:...:c0 -j DROP
\paragraph{3) Protezione della risoluzione indirizzi.}

\textbf{DHCP Snooping e Dynamic ARP Inspection (DAI).}
Il \textbf{DHCP Snooping} è una funzione che impedisce agli attaccanti di impersonare server DHCP. Registrando le informazioni IP–MAC–porta–VLAN provenienti dalle risposte DHCP legittime, il dispositivo protegge contro il \textbf{DHCP spoofing}. L'integrazione con \textbf{Dynamic ARP Inspection} (DAI) fornisce un ulteriore livello di protezione, assicurando che solo le risposte ARP coerenti con la tabella DHCP siano accettate. Questo blocca attacchi come \emph{ARP poisoning}, impedendo che il traffico venga dirottato verso un attaccante.

\textbf{SEND (Secure Neighbor Discovery) per IPv6.}
Per le reti IPv6, la protezione contro gli attacchi di tipo \emph{neighbor spoofing} o \emph{router advertisement flooding} è fornita da \textbf{SEND (Secure Neighbor Discovery)}. Questo protocollo estende NDP (Neighbor Discovery Protocol) e garantisce l’autenticità delle comunicazioni grazie all’utilizzo di \textbf{CGA (Cryptographically Generated Addresses)} e firme digitali.
SEND è utile per prevenire che un attaccante falsifichi i messaggi NDP e si faccia passare per un router o un altro host legittimo.

\paragraph{4) Crittografia L2: MACsec (802.1AE).}

Il protocollo \textbf{MACsec (802.1AE)} fornisce la crittografia end-to-end del traffico a livello L2, garantendo \emph{confidenzialità}, \emph{integrità} e protezione contro gli \emph{attacchi replay}. Con \textbf{GCM-AES-128} come cifratura predefinita, MACsec protegge i frame Ethernet, inclusi quelli che attraversano domini di trasmissione non protetti.
Un esempio pratico di configurazione con \texttt{ip link} in Linux è il seguente: 
# Esempio di configurazione MACsec su Linux
ip link add link eth0 macsec0 type macsec
ip macsec add macsec0 tx sa 0 pn 1 on key 01 0987...32109
ip macsec add macsec0 rx address b0:...:b0 port 1 sa 0 pn 1 on key 02 1234...9012
ip link set macsec0 up
ip addr add 10.100.0.2/24 dev macsec0

Limiti: MACsec non risolve i problemi relativi agli attacchi DoS (Denial of Service), né impedisce a un host legittimo compromesso di compromettere ulteriormente la rete.

\paragraph{5) Buone pratiche operative.}

Oltre alle soluzioni tecnologiche, le buone pratiche operative sono fondamentali.
Le seguenti azioni devono essere adottate in modo sistematico:
\begin{itemize}
\item Protezione fisica: assicurarsi che l'hardware sia protetto in armadi blindati e che le porte non utilizzate siano disabilitate.
\item Shutdown delle porte non usate: evitare che porte inutilizzate possano essere sfruttate per un "join fisico" non autorizzato.
\item Storm Control: attivare meccanismi di controllo dei flussi di traffico per evitare tempeste di broadcast.
\item BPDU Guard / Root Guard: proteggere gli switch da manipolazioni del protocollo STP.
\item Principio del least privilege: limitare i privilegi amministrativi e configurare correttamente gli accessi remoti.
\item Logging e monitoraggio continuo: per rilevare eventuali anomalie e ridurre i tempi di reazione agli attacchi.
\end{itemize}

\subsubsection{Monitoraggio e risposta}
\textbf{Firewall & DPI (Deep Packet Inspection).}
I firewall moderni, operanti su tutti i livelli, permettono di controllare il traffico tra segmenti e di eseguire analisi di pacchetti (DPI) per identificare eventuali attacchi. Utilizzando \texttt{ebtables}, \texttt{iptables} e \texttt{netfilter}, è possibile applicare ACL a livello L2 e superiore.

\textbf{IDS/IPS (Intrusion Detection / Prevention Systems).}
Questi sistemi analizzano il traffico di rete alla ricerca di pattern noti di attacco, utilizzando librerie di firme. Vengono solitamente collocati tra due endpoint per monitorare il traffico tramite port mirroring.

\textbf{eBPF/XDP (Extended Berkeley Packet Filter / Express Data Path).}
Questi strumenti ad alte prestazioni permettono di applicare filtri sui pacchetti direttamente nel kernel del sistema operativo, consentendo una gestione avanzata del traffico per il rate-limiting, il rilevamento di spoofing e la telemetria avanzata.

\subsubsection{Appendice A — Dalla LAN agli overlay: EVPN/VXLAN}
In scenari di overlay L2 su L3, come VXLAN, il traffico di broadcast viene incapsulato nei tunnel. \textbf{EVPN} (Ethernet Virtual Private Network) introduce un control-plane BGP per la gestione dinamica dei MAC e IP, riducendo la necessità di ARP e migliorando la resilienza del sistema contro gli attacchi basati sul flooding.

\subsubsection{Appendice B — Checklist operativa}
\begin{itemize}
\item Segmenta la rete con router/firewall e VLAN.
\item Usa 802.1X per l'autenticazione delle porte utente.
\item Abilita DHCP Snooping + Dynamic ARP Inspection + IP Source Guard.
\item Applica Port Security e ACL mirate.
\item Considera MACsec per proteggere la confidenzialità dei dati.
\item Monitora il traffico con IDS/IPS e applica un logging rigoroso.
\end{itemize}

%%%%%%%%%%%%%%%%%%%%%%%%%%%%%%sezione 3
\newpage
\maketitle
\section{NET\_03 — Virtual LANs (VLAN)}



\subsubsection{Definizione e motivazione}
Gli \textbf{switch Ethernet} tradizionali segmentano i domini di collisione ma non i domini di broadcast. Ciò significa che tutti gli host connessi a uno switch appartengono allo stesso dominio di broadcast e ricevono tutti i frame inviati in broadcast (come richieste ARP o DHCP).  

Questa architettura è semplice ma poco scalabile: in reti di grandi dimensioni, il traffico broadcast può saturare la banda disponibile e ogni problema (come un loop o un attacco) può propagarsi a tutti gli host della rete.

Per risolvere questo limite si introduce il concetto di \textbf{Virtual LAN (VLAN)}, cioè la creazione di \emph{più domini di broadcast logici} all’interno della stessa infrastruttura fisica. Ogni VLAN rappresenta una “rete virtuale” indipendente, isolata logicamente dalle altre pur condividendo gli stessi apparati.

\subsubsection{Limiti delle reti fisicamente separate}
Storicamente, la separazione dei domini di broadcast veniva realizzata con \textbf{sottoreti IP fisiche}, ciascuna connessa a un proprio switch e router. Tuttavia questo approccio presenta diversi limiti:
\begin{itemize}
  \item necessità di cablaggi distinti e di apparati separati per ogni subnet anche se gli switch sono sullo stesso piano (fisico) come mostrato in Figura~\ref{fig:ip_subnet};
  \item difficoltà di riconfigurazione in caso di spostamento di host tra subnet diverse;
  \item costi di gestione e manutenzione elevati.
\end{itemize}

\begin{figure}[H]
    \centering
    \includegraphics[width=.85\linewidth]{immagini/NET_03/ip_subnet.png}
    \caption{Physical ip subnet}
    \label{fig:ip_subnet}
\end{figure}


Con l’introduzione degli \textbf{switch di Layer 3}, la velocità di routing non è più un problema, ma resta la necessità di una gestione logica più flessibile. Le VLAN forniscono questa flessibilità permettendo di isolare logicamente i gruppi di dispositivi in base a criteri funzionali, e non fisici.

\paragraph{Benefici principali delle VLAN}
\begin{itemize}
  \item \textbf{Confinamento del broadcast:} il traffico broadcast resta confinato all’interno della VLAN di appartenenza.
  \item \textbf{Scalabilità e ordine:} la rete può essere gestita come un insieme di domini separati, semplificando la diagnostica.
  \item \textbf{Sicurezza:} la separazione logica riduce la superficie d’attacco e impedisce la propagazione di minacce L2 tra gruppi diversi.
\end{itemize}


\begin{figure}[H]
    \centering
    \includegraphics[width=0.85\linewidth]{immagini/NET_03/vlan.png}
    \caption{VLAN=area che limita il broadcast domain}
    \label{fig:vlan}
\end{figure}


\subsection{Assegnazione e membership delle VLAN}

\subsubsection{Criteri di assegnazione}
Un dispositivo può essere assegnato a una VLAN in base a diversi criteri:

\begin{itemize}
  \item \textbf{Per porta (Port-based VLAN):} lo switch associa staticamente una VLAN a ogni porta. È il metodo più comune e lo standard definito da IEEE~802.1Q.
  \item \textbf{Per MAC address (User-based VLAN):} la VLAN è determinata dal MAC del dispositivo o dall’identità dell’utente autenticato (es. via 802.1X).
  \item \textbf{Per protocollo (Protocol-based VLAN):} introdotto da IEEE~802.1v, assegna la VLAN in base al protocollo di livello 3 (IP, IPX, ecc.).
  \item \textbf{Combinato (Cross-layer):} alcune implementazioni permettono regole gerarchiche, ad esempio prima per protocollo, poi per MAC, e infine per porta.
\end{itemize}

\subsubsection{Vista logica}
Ogni VLAN rappresenta un dominio di broadcast indipendente e, di conseguenza, è normalmente associata a una \textbf{sottorete IP dedicata}.  
La comunicazione tra VLAN diverse richiede un dispositivo L3 (router o Layer 3 switch (figura ~\ref{fig:vlan2})) che svolga la funzione di \emph{inter-VLAN routing}.

\begin{figure}[H]
    \centering
    \includegraphics[width=0.85\linewidth]{immagini/NET_03/vlan2.png}
    \caption{Physical vs logical view}
    \label{fig:vlan2}
\end{figure}

\paragraph{Router “one-armed”}  
In una configurazione one-armed router, una \textbf{singola interfaccia fisica} del router viene utilizzata per gestire \textbf{più VLAN} contemporaneamente.  
Ciò avviene tramite la creazione di \textbf{sub-interfacce virtuali}, ognuna associata a una VLAN specifica.  

\begin{figure}[H]
    \centering
    \includegraphics[width=0.65\linewidth]{immagini/NET_03/vlan3.png}
    \caption{VLAN e sottoreti IP collegate tramite sub-interfacce}
    \label{fig:vlan3}
\end{figure}

\begin{lstlisting}[language=bash,caption={Esempio di configurazione router one-armed}]
ip link add link eth0 name eth0.10 type vlan id 10
ip link add link eth0 name eth0.20 type vlan id 20
ip addr add 10.0.10.1/24 dev eth0.10
ip addr add 10.0.20.1/24 dev eth0.20
\end{lstlisting}

In questo esempio, l’interfaccia fisica \texttt{eth0} viene suddivisa in due sub-interfacce virtuali, 
\texttt{eth0.10} e \texttt{eth0.20}, rispettivamente associate alle VLAN 10 e 20.  
A ciascuna sub-interfaccia viene assegnato un indirizzo IP appartenente alla sottorete della relativa VLAN.  

In questo modo, il router può fornire servizi di \textbf{routing inter-VLAN}, agendo come \textbf{gateway predefinito} 
per ciascuna rete logica, pur utilizzando una sola porta fisica.  
Tale approccio è comune nei laboratori e negli ambienti di test, dove si vuole semplificare la topologia 
riducendo il numero di interfacce fisiche necessarie.



\subsection{Trasporto dei frame: Tagging IEEE 802.1Q}

Lo standard \textbf{IEEE 802.1Q} permette di far convivere più VLAN sullo stesso collegamento fisico.
Per farlo, inserisce all’interno dei frame Ethernet un piccolo campo aggiuntivo, chiamato \emph{tag VLAN}, 
che identifica la VLAN di appartenenza del frame.  
A seconda del tipo di collegamento, gli switch 802.1Q gestiscono questi tag in modo diverso, distinguendo tre tipologie di porte.

\subsubsection{Porte di tipo Access, Trunk e Hybrid}

\begin{itemize}
  \item \textbf{Access Port:} collega un host finale (ad esempio un PC o una stampante).  
  I frame che transitano su una porta di questo tipo sono sempre \emph{senza tag} (\emph{untagged}).  
  Lo switch associa internamente la porta a una specifica VLAN, aggiungendo o rimuovendo automaticamente il tag durante il transito interno.

  \item \textbf{Trunk Port:} collega due apparati di rete (ad esempio switch–switch o switch–router).  
  Trasporta frame appartenenti a più VLAN contemporaneamente, inviandoli e ricevendoli \emph{con tag} 802.1Q.  
  In questo modo, ciascun frame mantiene l’informazione sulla VLAN di origine anche attraverso un link condiviso.

  \item \textbf{Hybrid Port:} gestisce sia frame \emph{taggati} che \emph{non taggati}.  
  I frame non taggati vengono automaticamente associati alla \textbf{Native VLAN}, mentre quelli taggati mantengono il proprio VLAN ID.  
  Questo tipo di porta è utile quando si collegano dispositivi che supportano il tagging VLAN insieme ad altri che non lo supportano.
\end{itemize}


\subsubsection{Access links}

Un \textbf{Access link} è un collegamento che parte da una \textbf{porta di tipo Access}, 
ossia una porta configurata per appartenere a una singola VLAN.  
Questo tipo di collegamento è utilizzato per connettere dispositivi finali (come PC, stampanti o server) 
oppure piccoli hub o switch non gestiti.

I frame che transitano su una porta di tipo Access sono sempre \emph{non taggati} (\emph{untagged}):  
gli host collegati inviano e ricevono normali frame Ethernet, senza alcuna informazione sulla VLAN.  
È lo switch che, internamente, associa la porta a una VLAN e gestisce l’inserimento o la rimozione del \emph{tag 802.1Q} 
quando il frame entra o esce dalla rete VLAN-aware.  

In questo modo, i dispositivi connessi non devono essere consapevoli dell’esistenza delle VLAN:  
dal loro punto di vista fanno parte semplicemente di una rete Ethernet dedicata, tipicamente 
corrispondente a una specifica sottorete IP.

\begin{figure}[H]
    \centering
    \includegraphics[width=0.25\linewidth]{immagini/NET_03/access_link.png}
    \caption{Esempio di collegamento Access link tra host e switch VLAN-aware}
    \label{fig:access_link}
\end{figure}

\subsubsection{Access links nelle regioni legacy}

In alcuni contesti, detti \textbf{legacy regions}, un access link può estendersi attraverso 
piccole LAN composte da più switch che non supportano le VLAN (\emph{VLAN-unaware switches}).  
In questi casi, l’intera rete tradizionale viene vista dallo switch VLAN-aware come un unico segmento Ethernet 
appartenente a una sola VLAN.  

Tutti i dispositivi collegati all’interno di tale regione condividono la stessa VLAN, 
anche se gli switch intermedi non gestiscono i tag 802.1Q.  
Questo approccio consente di integrare reti esistenti non VLAN-aware in un’infrastruttura moderna basata su VLAN.

\begin{figure}[H]
    \centering
    \includegraphics[width=0.85\linewidth]{immagini/NET_03/access_link_legacy.png}
    \caption{Access link esteso in una regione legacy con switch non VLAN-aware}
    \label{fig:access_link_legacy}
\end{figure}


\subsubsection{Trunk links}

Un \textbf{Trunk link} è un collegamento che parte da una \textbf{porta di tipo Trunk}, 
utilizzato per trasportare frame appartenenti a più VLAN contemporaneamente.  
È tipicamente impiegato nei collegamenti \emph{switch–switch} o \emph{switch–router}, 
dove è necessario far transitare traffico di più reti logiche sullo stesso mezzo fisico.

A differenza delle porte Access, le porte Trunk trasmettono e ricevono \textbf{frame taggati} con l’identificatore VLAN 
secondo lo standard \textbf{IEEE 802.1Q}.  
Il tag 802.1Q consente di distinguere a quale VLAN appartiene ciascun frame, 
evitando la confusione tra traffici di reti diverse che condividono il link.

Un trunk link \textbf{non appartiene direttamente a una VLAN}, ma può trasportare:
\begin{itemize}
  \item frame provenienti da \emph{tutte} le VLAN configurate sullo switch;
  \item oppure frame appartenenti solo a un sottoinsieme di VLAN selezionate.
\end{itemize}


\begin{figure}[H]
    \centering
    \includegraphics[width=0.6\linewidth]{immagini/NET_03/trunk_link.png}
    \caption{Esempio di collegamento Trunk tra apparati VLAN-aware}
    \label{fig:trunk_link}
\end{figure}


\subsubsection{Hybrid links}

Le \textbf{Hybrid links} rappresentano un’evoluzione dei trunk link e supportano sia 
\textbf{frame taggati} sia \textbf{frame non taggati}.  
I frame taggati mantengono il proprio VLAN ID, mentre i frame non taggati vengono associati 
a una VLAN predefinita (la \textbf{Native VLAN}).

Questo tipo di collegamento è utile quando sullo stesso link devono transitare:
\begin{itemize}
  \item traffici di apparati VLAN-aware (che utilizzano frame taggati);
  \item e traffici di dispositivi legacy o VLAN-unaware (che inviano frame non taggati).
\end{itemize}

In sostanza, un hybrid link permette di far convivere traffico VLAN multiplo e traffico Ethernet standard
sullo stesso collegamento, garantendo compatibilità tra apparati di diversa generazione.  
Nelle implementazioni moderne, molti switch trattano di fatto tutti i link come \emph{ibridi}, 
in grado di gestire dinamicamente entrambe le tipologie di frame.

\begin{figure}[H]
    \centering
    \includegraphics[width=0.75\linewidth]{immagini/NET_03/hybrid_link.png}
    \caption{Esempio di Hybrid link che trasporta frame taggati e non taggati}
    \label{fig:hybrid_link}
\end{figure}
\smallskip
In figura si nota come il collegamento ibrido consenta il transito simultaneo di traffico proveniente da 
VLAN diverse, includendo anche host non VLAN-aware (VLAN~C), senza compromettere la separazione logica 
delle altre VLAN taggate.


\subsubsection{Una stazione può appartenere a più VLAN?}

In generale, un host connesso tramite una \textbf{Access port} appartiene a una sola VLAN, 
poiché i frame che transitano su quella porta sono sempre \emph{non taggati} e associati a una singola rete logica.

Tuttavia, è possibile che una stessa stazione appartenga a \textbf{più VLAN} contemporaneamente.  
Questo avviene quando la stazione dispone di una \textbf{interfaccia trunk}, in grado di inviare e ricevere 
\emph{frame taggati 802.1Q} appartenenti a VLAN diverse.

Un caso tipico è quello dei \textbf{server multi-VLAN}, che devono comunicare con più reti logiche 
(o sottoreti IP) attraverso un’unica interfaccia fisica.  
In tali scenari, il sistema operativo del server crea \textbf{sub-interfacce virtuali} 
(es. \texttt{eth0.10}, \texttt{eth0.20}), ciascuna configurata con un VLAN ID differente, 
consentendo la separazione logica del traffico pur utilizzando la stessa scheda di rete.

\begin{figure}[H]
    \centering
    \includegraphics[width=0.75\linewidth]{immagini/NET_03/multivlan_station.png}
    \caption{Esempio di stazione connessa a più VLAN tramite interfaccia trunk}
    \label{fig:multivlan_station}
\end{figure}

In sintesi, solo le stazioni dotate di interfacce \emph{VLAN-aware} possono appartenere a più VLAN: 
sono esse a gestire il tagging e l’instradamento del traffico tra le diverse reti logiche.


\newpage
\maketitle
{\LARGE \textbf{Laboratorio}} \\[0.5em]
\subsection{Laboratorio 3: Configurazione VLAN e router one-armed}

In questo laboratorio si analizza il funzionamento delle VLAN e del \textbf{routing inter-VLAN} 
attraverso un router connesso tramite una singola interfaccia fisica (\emph{one-armed router}).  
L’obiettivo è comprendere come i frame vengano trasportati attraverso collegamenti di tipo 
\emph{Access}, \emph{Trunk} e \emph{Hybrid} secondo lo standard IEEE 802.1Q.

\subsection*{Topologia di rete}
La rete è composta da:
\begin{itemize}
  \item uno \textbf{switch VLAN-aware} con tre porte configurate come Access (VLAN 10 e 20) 
        e una porta configurata come Trunk verso il router;
  \item due host collegati alle porte Access, ciascuno appartenente a una VLAN distinta;
  \item un router configurato con sub-interfacce virtuali (\texttt{eth0.10}, \texttt{eth0.20}) 
        per fornire connettività inter-VLAN.
\end{itemize}

\begin{figure}[H]
    \centering
    \includegraphics[width=0.75\linewidth]{immagini/NET_03/lab3_topology.png}
    \caption{Topologia del Laboratorio 3: VLAN e router one-armed}
    \label{fig:lab3_topology}
\end{figure}

\subsection*{Configurazione di esempio}
\begin{lstlisting}[language=bash,caption={Configurazione delle VLAN sul router one-armed}]
# Creazione delle sub-interfacce VLAN
ip link add link eth0 name eth0.10 type vlan id 10
ip link add link eth0 name eth0.20 type vlan id 20

# Assegnazione degli indirizzi IP (gateway delle rispettive VLAN)
ip addr add 10.0.10.1/24 dev eth0.10
ip addr add 10.0.20.1/24 dev eth0.20

# Attivazione delle interfacce
ip link set eth0.10 up
ip link set eth0.20 up
\end{lstlisting}

Sul lato switch:
\begin{lstlisting}[language=bash,caption={Configurazione delle VLAN sullo switch}]
# Creazione VLAN
vlan 10
vlan 20

# Assegnazione delle porte
interface eth1
 switchport mode access
 switchport access vlan 10

interface eth2
 switchport mode access
 switchport access vlan 20

# Porta trunk verso il router
interface eth0
 switchport mode trunk
 switchport trunk allowed vlan 10,20
\end{lstlisting}
\subsection*{Funzionamento del laboratorio}

Il laboratorio ha lo scopo di mostrare in modo pratico il funzionamento delle \textbf{VLAN} 
e del \textbf{router one-armed}, ovvero una configurazione in cui un unico collegamento fisico 
tra router e switch trasporta il traffico di più VLAN tramite \textbf{frame taggati IEEE~802.1Q}.

\paragraph{Isolamento del traffico}
Gli host collegati alle \textbf{porte Access} appartengono ciascuno a una VLAN distinta.  
I frame che transitano su queste porte sono \emph{non taggati} e vengono separati dallo switch in base alla VLAN di appartenenza.  
In questo modo, gli host di VLAN diverse non possono comunicare direttamente: lo switch isola i domini di broadcast 
e impedisce la comunicazione a livello 2.

\paragraph{Routing inter-VLAN}
Per permettere la comunicazione tra VLAN diverse, il traffico deve passare attraverso il router.  
La connessione tra router e switch avviene tramite una \textbf{porta di tipo Trunk}, che trasporta i frame di più VLAN 
aggiungendo un tag 802.1Q a ciascun frame.  
Sul router, l'interfaccia fisica (ad esempio \texttt{eth0}) è suddivisa in più \textbf{sub-interfacce virtuali} 
(\texttt{eth0.10}, \texttt{eth0.20}, ecc.), ognuna configurata con:
\begin{itemize}
  \item un \textbf{VLAN ID} specifico;
  \item un indirizzo IP che funge da \textbf{gateway} per la relativa VLAN.
\end{itemize}

Quando un host della VLAN~10 invia un pacchetto verso un host della VLAN~20:
\begin{enumerate}
  \item il frame raggiunge lo switch sulla porta Access e viene inoltrato sul trunk verso il router con tag VLAN~10;
  \item il router riceve il frame su \texttt{eth0.10}, lo elabora a livello 3 e decide di inoltrarlo sulla sub-interfaccia \texttt{eth0.20};
  \item il router rimanda il frame allo switch, questa volta con tag VLAN~20;
  \item lo switch rimuove il tag e lo invia alla porta Access corrispondente alla VLAN~20.
\end{enumerate}

\begin{figure}[H]
    \centering
    \includegraphics[width=0.85\linewidth]{immagini/NET_03/example.png}
    \caption{Comunicazione tra diverse vlan}
    \label{fig:example}
\end{figure}


\paragraph{Verifica del comportamento}
Nel laboratorio si può verificare che:
\begin{itemize}
  \item gli host della stessa VLAN comunicano direttamente a livello~2, senza passare dal router;
  \item la comunicazione tra VLAN diverse avviene tramite il router (routing inter-VLAN);
  \item tutto il traffico tra router e switch è trasportato sul link trunk mediante frame taggati 802.1Q.
\end{itemize}

\paragraph{Osservazione pratica}
Catturando il traffico con \texttt{tcpdump} o \texttt{Wireshark} sull’interfaccia trunk, 
è possibile osservare i \textbf{tag VLAN} all’interno dei frame Ethernet.  
Ciò consente di verificare visivamente il meccanismo di separazione e instradamento del traffico 
tra le diverse VLAN.




\newpage
\subsection{Sicurezza delle VLAN e vulnerabilità di livello 2}

Le VLAN migliorano l’isolamento logico del traffico, ma non eliminano le minacce 
presenti a livello di collegamento.  
Un attaccante connesso alla rete locale può sfruttare debolezze dei protocolli 
di livello~2 (Ethernet e ARP) o configurazioni errate degli switch per intercettare,
modificare o dirottare il traffico di rete.

\subsubsection{Minacce principali}

\paragraph{1) MAC Flooding (CAM Overflow)}
Gli switch mantengono in memoria una \textbf{Content Addressable Memory (CAM)}, 
che associa indirizzi MAC a porte fisiche.  
Un attaccante può inviare migliaia di frame con indirizzi MAC falsi, 
riempiendo la tabella CAM e provocando un \emph{overflow}.  
Quando la tabella è satura, lo switch non riesce più a determinare su quale porta si trova un determinato MAC 
e inizia a inoltrare i frame in \textbf{broadcast}, esponendo il traffico all’attaccante.

\begin{figure}[H]
    \centering
    \includegraphics[width=0.75\linewidth]{immagini/NET_03/mac_attack.png}
    \caption{Esempio di MAC flooding}
    \label{fig:mac_attack}
\end{figure}

\emph{Mitigazione:} abilitare la \textbf{port security}, limitando il numero di MAC address appresi per ciascuna porta, 
e disattivare l’apprendimento dinamico dove non necessario.

\paragraph{2) ARP Spoofing / Poisoning}
Il protocollo ARP non prevede autenticazione, quindi un attaccante può inviare \textbf{risposte ARP falsificate} 
per associare il proprio MAC all’indirizzo IP del gateway o di altri host nella stessa VLAN.  
In questo modo, intercetta o altera il traffico tra due dispositivi (\emph{Man-in-the-Middle}).


\emph{Mitigazione:} configurare \textbf{ARP statici} per i dispositivi critici, utilizzare strumenti di monitoraggio 
come \texttt{arpwatch} o implementare sistemi di protezione come \texttt{Dynamic ARP Inspection (DAI)} sugli switch gestiti.

\paragraph{3) VLAN Hopping}
Questa categoria di attacchi consente a un host di inviare o ricevere traffico appartenente a una VLAN diversa 
da quella assegnata, violando l’isolamento logico.  
Le due tecniche principali sono:
\begin{itemize}
  \item \textbf{Basic VLAN hopping:} l’attaccante sfrutta il protocollo DTP (\emph{Dynamic Trunking Protocol}) 
        per negoziare automaticamente una connessione di tipo trunk con lo switch, ottenendo accesso a più VLAN.
    \begin{figure}[H]
        \centering
        \includegraphics[width=0.5\linewidth]{immagini/NET_03/basic_vlan_hopping.png}
        \caption{Collegamento tra due switch tramite porte trunk che trasportano il traffico di più VLAN sullo stesso link fisico.}
        \label{fig:vlan_hopper}
    \end{figure}
  \item \textbf{Double Tagging:} il frame viene costruito con due tag 802.1Q annidati.  
        Il primo tag (relativo alla VLAN nativa) viene rimosso dallo switch di ingresso, 
        lasciando il secondo, che identifica la VLAN bersaglio.
    \begin{figure}[H]
        \centering
        \includegraphics[width=0.5\linewidth]{immagini/NET_03/double_tagging.png}
        \caption{Esempio di attacco \textbf{VLAN Double Tagging}: un host malevolo appartenente alla VLAN~10 inserisce due tag 802.1Q (VLAN~10 e VLAN~40). 
    Il primo switch rimuove il tag della VLAN~nativa (10) e inoltra il frame, che conserva il secondo tag (40). 
    Il frame attraversa quindi il trunk ed entra nella VLAN~40, violando l’isolamento tra VLAN.}
        \label{fig:double_tagging}
    \end{figure}
\end{itemize}

\emph{Mitigazione:} disabilitare DTP e configurare manualmente le porte come \textbf{Access} dove necessario;  
evitare l’uso della \textbf{VLAN~1} come VLAN nativa e assegnare VLAN native dedicate non utilizzate per il traffico utente.
\paragraph{4) Attacchi ai protocolli di controllo}

Oltre agli attacchi diretti alle VLAN, anche i protocolli di \textbf{controllo e gestione} 
utilizzati dagli switch di livello~2 possono essere sfruttati da un attaccante per alterare 
la topologia della rete o raccogliere informazioni sensibili.  
Tra i principali:


\subparagraph{Spanning Tree Attack (BPDU spoofing)}
\begin{itemize}
  \item \textbf{Cosa fa lo STP:} gli switch si scambiano BPDU (bridge protocol data units) per eleggere il \emph{root bridge} e stabilire quali porte siano forwarding o blocking, evitando loop.
  \item \textbf{Come attacca l'avversario:} l'attaccante invia BPDU falsi con una priorità molto bassa (o con un bridge ID fittizio), facendo credere agli switch che il suo dispositivo sia il nuovo root.
  \item \textbf{Effetto pratico:} la topologia si ricalcola; alcuni link possono essere forzati in forwarding creando percorsi non previsti, perdita di connettività o instradamento del traffico attraverso il nodo dell'attaccante.
  \item \textbf{Mitigazione:} \emph{BPDU Guard} spegne la porta se riceve BPDU su una porta che dovrebbe essere una porta access (cioè verso host), mentre \emph{Root Guard} impedisce a un certo segmento di diventare root forzando il comportamento previsto.
\end{itemize}

\subparagraph{VTP Attack (VLAN Trunking Protocol)}
\begin{itemize}
  \item \textbf{Cosa fa VTP:} permette di distribuire automaticamente la lista delle VLAN a tutti gli switch del dominio VTP.
  \item \textbf{Come attacca l'avversario:} un dispositivo malevolo si presenta come \texttt{server VTP} con una \emph{revision number} superiore; gli altri switch accettano la nuova configurazione e sovrascrivono le VLAN locali.
  \item \textbf{Effetto pratico:} le VLAN possono essere cancellate o modificate su larga scala, causando indisponibilità o perdita dell'isolamento tra reti.
  \item \textbf{Mitigazione:} impostando VTP in \emph{transparent} o disabilitandolo si evita che uno switch accetti e propaghi automaticamente configurazioni provenienti da fonti non affidabili.
\end{itemize}

\subparagraph{Cisco Discovery Protocol Attack (information leakage)}
\begin{itemize}
  \item \textbf{Cosa fa CDP:} i dispositivi Cisco pubblicano informazioni (IP, modelli, versioni) su CDP verso i vicini.
  \item \textbf{Come attacca l'avversario:} un host malintenzionato cattura i pacchetti CDP o ascolta il traffico e ottiene dettagli utili per attacchi mirati (es. versioni vulnerabili).
  \item \textbf{Effetto pratico:} ricognizione facilitata: l'attaccante conosce quali dispositivi e software colpire.
  \item \textbf{Mitigazione:} disabilitando CDP sulle porte utenti si riduce la quantità di informazioni esposte ai terminali non fidati.
\end{itemize}






%%%%%%%%%%%%%%%%%       laboratorio         %%%%%%%%%%
\newpage
\maketitle
{\LARGE \textbf{Laboratorio}} \\[0.5em]
\subsection{Laboratorio 4: VLAN Hopping e Double Tagging}

\subsubsection{Scenario}
L’obiettivo del laboratorio è simulare un attacco di \textbf{Double Tagging}, in cui un host appartenente alla VLAN nativa tenta di raggiungere host di un’altra VLAN senza passare per il router.

\begin{figure}[H]
  \centering
  \includegraphics[width=.8\linewidth]{immagini/NET_03/double_tagging.png}
  \caption{Attacco Double Tagging su trunk VLAN nativa}
  \label{fig:double_tagging}
\end{figure}

\subsubsection{Configurazione di attacco}
\begin{lstlisting}[language=bash,caption={Esempio pratico di doppio tagging}]
ip link add link eth0 name eth0.1 type vlan id 1
ip link set eth0.1 up
ip link add link eth0.1 name eth0.1.20 type vlan id 20
ip link set eth0.1.20 up
ip addr add 10.0.20.250/24 dev eth0.1.20
arp -s 10.0.20.102 <MAC-vittima> -i eth0.1.20
ping 10.0.20.102
\end{lstlisting}

Questo attacco è tipicamente \textbf{unidirezionale}: l’attaccante può inviare pacchetti verso la VLAN vittima, ma non ricevere risposte.

\paragraph{Mitigazioni}
\begin{itemize}
  \item disabilitare l’auto-trunking (DTP) sulle porte utente;
  \item non usare VLAN~1 come nativa;
  \item assegnare VLAN nativa diversa e dedicata solo ai trunk;
  \item monitorare traffico anomalo su frame doppiamente taggati.
\end{itemize}


\subsection{Autenticazione e controllo d’accesso: IEEE 802.1X}

\subsubsection{Architettura e funzionamento}
IEEE~802.1X definisce un meccanismo di autenticazione basato su porta.  
Il modello coinvolge tre componenti:
\begin{itemize}
  \item \textbf{Supplicant:} l’host che richiede accesso alla rete;
  \item \textbf{Authenticator:} lo switch o access point che funge da intermediario;
  \item \textbf{Authentication Server:} tipicamente un server RADIUS che verifica le credenziali.
\end{itemize}

Il protocollo utilizza EAPOL (EAP over LAN) per trasportare le credenziali di autenticazione a livello 2, prima dell’assegnazione IP.  
Una volta autenticato, l’utente può essere automaticamente assegnato a una VLAN o a specifiche ACL in base all’identità verificata.

\paragraph{Vantaggi}
\begin{itemize}
  \item garantisce accesso controllato alle VLAN;
  \item permette la \textbf{VLAN dinamica per utente};
  \item si integra con firewall e sistemi NAC per gestione centralizzata.
\end{itemize}


\subsection{Firewall e sicurezza perimetrale}

\subsubsection{Ruolo del firewall}
I firewall rappresentano il principale punto di controllo tra VLAN e verso Internet.  
Esistono due principali categorie:
\begin{itemize}
  \item \textbf{Packet filtering:} filtra i pacchetti in base a indirizzi, porte e protocolli.
  \item \textbf{Stateful inspection:} tiene traccia dello stato delle connessioni (es. TCP SYN/ACK) consentendo solo traffico di risposta legittimo.
\end{itemize}

\paragraph{Implementazione in Linux}
Il framework \textbf{Netfilter} del kernel Linux implementa le funzioni di firewalling.  
Il comando \texttt{iptables} consente di creare regole di filtraggio e NAT per gestire le comunicazioni tra VLAN e verso l’esterno.

\begin{lstlisting}[language=bash,caption={Esempio di filtraggio stateful tra VLAN}]
# Permetti traffico dalla VLAN 10 alla VLAN 20
iptables -A FORWARD -i eth0.10 -o eth0.20 -m state --state ESTABLISHED,RELATED -j ACCEPT
iptables -A FORWARD -i eth0.20 -o eth0.10 -j ACCEPT
\end{lstlisting}


\subsection*{Conclusioni}
Le VLAN rappresentano un meccanismo fondamentale di segmentazione logica, riducendo il dominio di broadcast e migliorando ordine e sicurezza.  
Tuttavia, se configurate in modo errato, possono introdurre vulnerabilità specifiche (es. trunk non protetti o VLAN nativa mal gestita).  
Una corretta progettazione prevede:
\begin{itemize}
  \item configurazione manuale e controllata delle porte trunk;
  \item uso di VLAN dedicate per la gestione e di VLAN nativa distinta da quelle operative;
  \item integrazione con 802.1X, ACL e firewall per garantire isolamento e autenticazione a ogni livello.
\end{itemize}

%%%%%%%%%%%%%%%%%%%%%%%% NET_04 %%%%%%%%%%%%%%%%%%%%%%%%%%


\end{document}
