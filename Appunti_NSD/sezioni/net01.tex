\section{NET\_01}
\subsection{Architettura di base e principi di rete}

\subsubsection{Definizione e interconnessione}
Internet è un'\emph{inter-rete}: un insieme di numerose sotto-reti eterogenee connesse tra loro. Le sotto-reti possono essere basate su tecnologie diverse al Livello 2 (L2) — ad esempio 802.11 (WiFi), 802.3 (Ethernet), tecnologie cellulari (3G/4G), fibra ottica o ADSL — ma comunicano grazie a uno stack di protocolli comune, implementato sopra i diversi livelli fisici e MAC, nella logica del paradigma OSI. Il protocollo di base che abilita l'interoperabilità è l'Internet Protocol (IP), affiancato da protocolli di livello superiore come TCP, UDP e protocolli applicativi (per es. DNS, HTTP).

\subsubsection{Indirizzamento e instradamento}
Ogni dispositivo in una rete IP possiede un identificatore numerico univoco: l'indirizzo IP (32 bit per IPv4, 128 bit per IPv6). Le sotto-reti sono connesse mediante router, dispositivi che effettuano l'inoltro (forwarding) dei pacchetti dalla sorgente alla destinazione seguendo regole presenti nelle tabelle di instradamento.

Il forwarding si basa sul principio del \emph{Longest Prefix Match} (LPM): per decidere quale voce della routing table utilizzare si seleziona la corrispondenza con il prefisso di rete più lungo che include l'indirizzo di destinazione. L'inoltro può essere:
\begin{itemize}
  \item \emph{diretto}, quando la destinazione si trova nella stessa rete locale;
  \item \emph{indiretto}, quando la destinazione è raggiungibile tramite un next hop (salto successivo).
\end{itemize}

\begin{figure}[H]
  \centering
  \includegraphics[width=.85\linewidth]{immagini/slide1/stessa_rete.png}
  \caption{Instradamento diretto: sorgente e destinazione nella stessa rete locale (L2/L3).}
  \label{fig:stessa-rete}
\end{figure}

Il compito di IP è consegnare il pacchetto alla rete finale; la consegna al dispositivo specifico è rimandata al livello L2, che si occupa della mappatura IP <-> L2 (per esempio tramite ARP per IP <-> MAC).

\subsubsection{Anatomia dell'indirizzo IP e subnetting}
Le reti IP sono suddivise in subnet logiche. Due host appartenenti alla stessa subnet condividono i primi $X$ bit dell'indirizzo (la \emph{parte di rete}), mentre i restanti 32 $X$ bit identificano l'host. Dal 1984 si usa CIDR (Classless Inter Domain Routing): il prefisso non è più implicito per classi fisse ma viene specificato tramite una \emph{subnet mask} (o notazione ``/length''), che indica quali bit rappresentano la parte di rete. L'\textbf{i}-esimo bit della subnet mask è settato a \textbf{0} se i'\textbf{i}-esimo bit è nella host part; \textbf{1} se invece è nel prefisso network .Ad esempio, con indirizzo \texttt{192.168.1.12} e maschera \texttt{255.255.255.0} (ovvero \texttt{/24}) la rete è \texttt{192.168.1.0}.

\begin{figure}[H]
  \centering
  \includegraphics[width=.85\linewidth]{immagini/slide1/esempio_mask.png}
  \caption{Esempio network prefix}
  \label{fig:esempio_mask}
\end{figure}


Ogni rete ha due indirizzi ip \textbf{RISERVATI} ovvero:
\begin{itemize}
    \item\emph{Net address} tutti i bits nella parte host sono 0.
    \item\emph{Broadcast address} tutti i bits nella parte host sono 1.
\end{itemize}

\begin{figure}[H]
  \centering
  \includegraphics[width=.85\linewidth]{immagini/slide1/esempio_ip.png}
  \caption{trovare il network e broadcast address}
  \label{fig:esempio_ip}
\end{figure}

\subsubsection*{Esempio spiegato: trovare rete e broadcast di \texttt{209.85.129.99/27}}
 
Vogliamo ricavare l'indirizzo di \textbf{rete} e di \textbf{broadcast} per l'host \texttt{209.85.129.99/27}.

\paragraph{1) Cosa significa ``/27''}
La notazione \texttt{/27} dice che i \textbf{primi 27 bit} dell'indirizzo sono di \emph{rete} e i rimanenti sono di \emph{host}.
Nel formato ``a ottetti'', i primi 24 bit coincidono con i primi tre ottetti; quindi la /27 ``entra'' nel \textbf{quarto ottetto}:
\[
\underbrace{\text{[8] [8] [8]}}_{\text{rete}} \; \underbrace{\text{[3]}}_{\text{rete}} \; \underbrace{\text{[5]}}_{\text{host}}
\]
La maschera corrispondente è \texttt{255.255.255.224}, perché nell'ultimo ottetto i 3 bit di rete valgono \texttt{11100000\textsubscript{2}} $= 224$.

\paragraph{2) Perché basta guardare l'ultimo ottetto}
Con /27 i primi tre ottetti (\texttt{209.85.129}) restano identici per rete/host/broadcast.
Tutta la partizione in sottoreti avviene nel \textbf{quarto ottetto}. Qui rimangono \textbf{5 bit di host} $\Rightarrow$ ogni sottorete ha $2^5=32$ indirizzi contigui (un ``blocco'').

\paragraph{3) Trova il blocco in cui cade 99}
I blocchi nel quarto ottetto sono a passi di 32: \texttt{0--31}, \texttt{32--63}, \texttt{64--95}, \texttt{96--127}, \texttt{128--159}, \dots
Poiché \texttt{99} appartiene a \texttt{96--127}, la nostra \textbf{rete} è \texttt{209.85.129.\underline{96}} e il \textbf{broadcast} sarà l'ultimo del blocco, \texttt{209.85.129.\underline{127}}. Verifichiamo con l'AND bit-a-bit.

\paragraph{4) Verifica con l'AND (ultimo ottetto)}.

\begin{lstlisting}
[basicstyle=\ttfamily\small,frame=single]
99   = 01100011
224  = 11100000   (maschera /27 nell'ultimo ottetto)
AND    --------
       01100000 = 96   --> indirizzo di rete (quarto ottetto)
\end{lstlisting}
Quindi l'indirizzo di \textbf{rete} è \texttt{209.85.129.96}.

\paragraph{5) Broadcast: tutti i bit host a 1}
Nel broadcast si \emph{mantengono} i 3 bit di rete e si mettono a 1 i 5 bit di host:
\[
\text{rete (ult.\ ottetto)} = 01100000 \quad + \quad 00011111 \; (\text{tutti i bit host a 1}) \;=\; 01111111 = 127.
\]
Dunque \textbf{broadcast} = \texttt{209.85.129.127}.

\paragraph{6) Intervallo host e conteggio}
Gli host validi sono i numeri compresi \emph{tra} rete e broadcast:
\[
\texttt{209.85.129.97} \; \text{fino a} \; \texttt{209.85.129.126},
\]
per un totale di $2^5 - 2 = 30$ host (si escludono rete e broadcast).






\subsubsection{Sistemi autonomi (AS) e routing globale}
L'inter-rete è organizzata in Autonomous Systems (AS), domini amministrativi che gestiscono internamente le proprie politiche di instradamento. All'interno di un AS si adottano Interior Gateway Protocols (IGP) come OSPF, IS-IS o RIP; lo scambio di rotte tra AS diversi avviene tramite l'Exterior Gateway Protocol più usato: BGP, che garantisce la raggiungibilità globale.

\subsubsection{La routing table e l'algoritmo di lookup}
La routing table contiene entry costituite tipicamente da: indirizzo di destinazione, maschera/netmask, next hop e interfaccia d'uscita. La funzione di lookup per un pacchetto $p$ itera le voci ordinate per lunghezza del prefisso e restituisce la voce $i$ tale che
\[
(p.daddr \;\&\; i.mask) = i.addr,
\]
dove \texttt{\&} è l'AND bit-a-bit; se non si trova alcuna corrispondenza il pacchetto viene scartato.

\subsection{II. Il viaggio del pacchetto: esempio di richiesta DNS}

\subsubsection{Topologia e hop}
Un pacchetto generato da un browser per risolvere un nome (ad esempio \texttt{www.google.com}) percorre una serie di hop: rete domestica (WiFi), access point/router, edge router dell'AS dell'utente, router di confine (border router), una sequenza di AS di transito e infine il data center che ospita il servizio DNS o il server web. Ogni tratto può utilizzare tecnologie e politiche diverse, e rappresenta un potenziale punto di vulnerabilità.

\begin{figure}[H]
  \centering
  \includegraphics[width=.85\linewidth]{immagini/slide1/dns.png}
  \caption{dal web browser al web server}
  \label{fig:dns}
\end{figure}

\subsubsection{Stack protocollare e incapsulamento}
La richiesta DNS attraversa gli strati della pila:

\begin{itemize}
  \item \textbf{Livello applicativo (DNS)}: genera la query di risoluzione ("dammi l'indirizzo ip di www.google.com").
  \item \textbf{Livello trasporto (UDP)}: aggiunge porta sorgente (es. 5000) e porta destinazione (53), oltre al checksum.
  \item \textbf{Livello rete (IP)}: inserisce indirizzi IP sorgente e destinazione (es. \texttt{10.0.0.100} e \texttt{85.18.200.200}), TTL, eventuale fragmentation.
  \item \textbf{Livello accesso (L2)}: incapsula il frame con indirizzi MAC del next hop; questi cambiano ad ogni salto.
\end{itemize}

Durante il percorso gli indirizzi IP rimangono costanti, mentre gli indirizzi MAC vengono aggiornati hop-by-hop. Meccanismi come ARP permettono la traduzione dinamica IP <-> MAC all'interno di una stessa rete locale.



\begin{figure}[H]
  \centering
  \includegraphics[width=.85\linewidth]{immagini/slide1/dns2.png}
  \caption{richiesta DNS (semplificata)}
  \label{fig:dns2}
\end{figure}



\subsection{III. Vulnerabilità intrinseche di IP e TCP}

I protocolli storici di Internet sono stati progettati principalmente per interoperabilità e scalabilità, non per la sicurezza. Questo ha lasciato diverse debolezze intrinseche.

\subsubsection{Identificazione, spoofing e non-ripudio}
Gli \textbf{identificatori di rete} (indirizzi IP e MAC) sono semplici stringhe binarie facilmente manipolabili: un mittente può generare pacchetti con sorgente falsata (\textbf{IP spoofing}) oppure modificare l'indirizzo sorgente di pacchetti che sta inoltrando. Questo fenomeno rende possibile, ad esempio, l'impersonificazione di server legittimi (un attaccante può inviare pacchetti che sembrano provenire da un DNS server affidabile). \textbf{IP non fornisce meccanismi di autenticazione dell'origine}: non esiste un modo intrinseco per dimostrare che l'indirizzo sorgente di un pacchetto corrisponda realmente al mittente fisico, provocando problemi di \emph{repudiation}.

\subsubsection{Confidenzialità}
\textbf{Il protocollo IP non cifra il payload né fornisce protezione contro l'intercettazione}: catturare e decodificare pacchetti su un segmento di rete è, in molti casi, semplice. Inoltre gli utenti non controllano l'intero percorso seguito dai pacchetti; anche fidandosi del proprio ISP, sono necessari fiducia e verifiche su tutti gli AS attraversati. Attacchi di route hijacking o route leaking possono alterare il percorso e compromettere la riservatezza.

\subsubsection{Integrità dei dati}
IP, TCP e UDP usano checksum per rilevare errori di trasmissione (header e payload), ma questi meccanismi non sono progettati come primitive di sicurezza: sono vulnerabili a manipolazioni intenzionali poiché basta ricalcolare il checksum dopo la modifica del pacchetto. Per esempio, il checksum IP è una semplice somma/XOR su parole dell'header e non offre garanzie contro un attaccante attivo.



\subsubsection{Packet replication e anti-replay}
A livello IP \textbf{non} esistono numeri di sequenza o marcatori univoci che identifichino inequivocabilmente un pacchetto in un flusso; il problema anti-replay è quindi in gran parte non risolto a questo livello. TCP fornisce numeri di sequenza, ma essi sono destinati alla gestione dell'affidabilità e dell'ordine, non all'autenticazione. Poiché tali numeri non sono protetti criptograficamente, possono essere predetti o spoofati in alcuni scenari, consentendo replay o session hijacking se non vengono adottate contromisure a livello superiore (ad esempio TLS).

\subsubsection{Insicurezza delle mappature dinamiche}
Molti servizi critici si basano su mappature dinamiche non progettate per la sicurezza: DNS (nomi→IP), ARP (IP→MAC), tabelle di bridging (MAC→porta), e la stessa routing table (destinazione→next hop). Implementazioni legacy, come il DNS non autenticato, permettono a un attaccante di fornire risposte fasulle: la risoluzione nome→IP non è intrinsecamente verificabile senza meccanismi come DNSSEC.

\newpage
\maketitle
{\LARGE \textbf{Laboratorio}} \\[0.5em]
\subsection{Laboratorio 1: MiTM e DNS spoofing}

\subsubsection{Obiettivo e scenario}
L'obiettivo è dirottare richieste HTTP non cifrate attraverso DNS spoofing e impersonificazione di un sito (es. \texttt{http://netgroup.uniroma2.it}). Lo scenario tipico prevede un attaccante nella stessa rete locale della vittima; il resolver della vittima è configurato su un DNS pubblico (per es. \texttt{8.8.8.8}). L'attacco combina:
\begin{enumerate}
  \item lo sfruttamento della mappatura IP <-> MAC (via ARP spoofing) per stabilire un MiTM;
  \item la manipolazione delle risposte DNS per risolvere il nome del sito bersaglio verso un indirizzo controllato dall'attaccante.
\end{enumerate}


\begin{figure}[H]
    \centering
    \includegraphics[width=.85\linewidth]{immagini/slide1/lab1.png}
    \caption{Topologia rete}
    \label{fig:lab1}
\end{figure}

\subsubsection{Fasi dell'attacco}
L'attacco tipico è articolato in quattro step principali:
\begin{enumerate}
  \item \textbf{STEP 1 – MiTM} (ARP spoofing): l'attaccante dissipa nelle cache ARP della vittima e del gateway risposte ARP falsificate in modo da farsi passare per entrambi e intercettare il traffico.
  \item \textbf{STEP 2 – Intercettazione della richiesta DNS}: una volta in posizione di MiTM, l'attaccante intercetta le query DNS emesse dalla vittima.
  \item \textbf{STEP 3 – Spoofing della risposta DNS}: l'attaccante risponde con una risoluzione fasulla per il dominio bersaglio, puntando a un IP di sua proprietà.
  \item \textbf{STEP 4 – Impersonificazione del sito web}: l'attaccante serve una copia del sito (ottenuta tramite mirroring) dall'IP di controllo, così che la vittima riceva contenuti apparentemente legittimi.
\end{enumerate}

\subsubsection{STEP 1: ARP spoofing (MiTM)}
L'attaccante invia risposte ARP non richieste (opcode 2) sia alla vittima che al default gateway:
\begin{itemize}
  \item alla vittima: un frame ARP unicast indirizzato al MAC della vittima (\texttt{bb:bb:bb:bb:bb:bb}) affermando che l'IP del router (\texttt{10.0.0.1}) corrisponde al MAC dell'attaccante (\texttt{aa:aa:aa:aa:aa:aa});
  \item al gateway: un frame ARP unicast indirizzato al MAC del router (\texttt{cc:cc:cc:cc:cc:cc}) affermando che l'IP della vittima (\texttt{10.0.0.100}) corrisponde al MAC dell'attaccante (\texttt{aa:aa:aa:aa:aa:aa}).
\end{itemize}
Ripetendo queste risposte periodicamente, l'attaccante mantiene la posizione di MiTM.

\subsubsection{STEP 2 \& 3: intercettazione e DNS spoofing}
Dopo aver stabilito il MiTM, l'attaccante può reindirizzare le richieste DNS verso la sua macchina:

\begin{lstlisting}[language=bash, caption={Esempio: regola iptables per reindirizzare richieste DNS (UDP 53) alla macchina locale}]
iptables -t nat -A PREROUTING -p udp --dport 53 -j REDIRECT
\end{lstlisting}

Sulla macchina dell'attaccante viene eseguito un server DNS leggero (es. \texttt{dnsmasq}) con una configurazione del tipo:

\begin{lstlisting}[caption={Estratto di /etc/dnsmasq.conf}]
interface=eth0
no-dhcp-interface=eth0
server=1.1.1.1

# Risolvi il dominio bersaglio verso l'IP dell'attaccante
address=/netgroup.uniroma2.it/10.0.0.200
\end{lstlisting}

Questa configurazione restituisce per \texttt{netgroup.uniroma2.it} l'indirizzo \texttt{10.0.0.200}; tutte le altre query vengono inoltrate al resolver pubblico (qui \texttt{1.1.1.1}).

\subsubsection{STEP 4: impersonificazione del sito web}
L'attaccante può aver replicato il contenuto del sito bersaglio tramite strumenti di mirroring, ad esempio:

\begin{lstlisting}[language=bash]
wget --mirror --convert-links --html-extension --no-parent -l 1 \
     --no-check-certificate http://netgroup.uniroma2.it
\end{lstlisting}

I contenuti mirrorati vengono serviti localmente (per es. con Apache2). Così la vittima, ricevendo l'IP dell'attaccante per il dominio richiesto, ottiene una copia apparentemente autentica del sito.

\subsection*{Conclusione e contromisure (sintesi)}
Le vulnerabilità descritte evidenziano che senza meccanismi di autenticazione, integrità e confidenzialità a livello superiore, l'infrastruttura IP/TCP è esposta a compromissioni. Contromisure pratiche includono:
\begin{itemize}
  \item utilizzo diffuso di canali cifrati e autenticati (TLS/HTTPS) per proteggere le applicazioni;
  \item adozione di estensioni e protocolli progettati per la sicurezza (es. DNSSEC per autenticare risposte DNS, IPsec per integrità/confidenzialità a livello IP dove applicabile);
  \item tecniche di difesa a livello di rete locale (ARP inspection, dynamic ARP protection, filtraggio di pacchetti spoofati sui router e access control lists);
  \item pratiche operative: aggiornamento dei software, monitoraggio delle anomalie di routing e validazione delle rotte BGP.
\end{itemize}
