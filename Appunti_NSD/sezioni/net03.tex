\newpage
\section{NET\_03 — Virtual LANs (VLAN)}



\subsubsection{Definizione e motivazione}
Gli \textbf{switch Ethernet} tradizionali segmentano i domini di collisione ma non i domini di broadcast. Ciò significa che tutti gli host connessi a uno switch appartengono allo stesso dominio di broadcast e ricevono tutti i frame inviati in broadcast (come richieste ARP o DHCP).  

Questa architettura è semplice ma poco scalabile: in reti di grandi dimensioni, il traffico broadcast può saturare la banda disponibile e ogni problema (come un loop o un attacco) può propagarsi a tutti gli host della rete.

Per risolvere questo limite si introduce il concetto di \textbf{Virtual LAN (VLAN)}, cioè la creazione di \emph{più domini di broadcast logici} all’interno della stessa infrastruttura fisica. Ogni VLAN rappresenta una “rete virtuale” indipendente, isolata logicamente dalle altre pur condividendo gli stessi apparati.

\subsubsection{Limiti delle reti fisicamente separate}
Storicamente, la separazione dei domini di broadcast veniva realizzata con \textbf{sottoreti IP fisiche}, ciascuna connessa a un proprio switch e router. Tuttavia questo approccio presenta diversi limiti:
\begin{itemize}
  \item necessità di cablaggi distinti e di apparati separati per ogni subnet anche se gli switch sono sullo stesso piano (fisico) come mostrato in Figura~\ref{fig:ip_subnet};
  \item difficoltà di riconfigurazione in caso di spostamento di host tra subnet diverse;
  \item costi di gestione e manutenzione elevati.
\end{itemize}

\begin{figure}[H]
    \centering
    \includegraphics[width=.85\linewidth]{immagini/NET_03/ip_subnet.png}
    \caption{Physical ip subnet}
    \label{fig:ip_subnet}
\end{figure}


Con l’introduzione degli \textbf{switch di Layer 3}, la velocità di routing non è più un problema, ma resta la necessità di una gestione logica più flessibile. Le VLAN forniscono questa flessibilità permettendo di isolare logicamente i gruppi di dispositivi in base a criteri funzionali, e non fisici.

\paragraph{Benefici principali delle VLAN}
\begin{itemize}
  \item \textbf{Confinamento del broadcast:} il traffico broadcast resta confinato all’interno della VLAN di appartenenza.
  \item \textbf{Scalabilità e ordine:} la rete può essere gestita come un insieme di domini separati, semplificando la diagnostica.
  \item \textbf{Sicurezza:} la separazione logica riduce la superficie d’attacco e impedisce la propagazione di minacce L2 tra gruppi diversi.
\end{itemize}


\begin{figure}[H]
    \centering
    \includegraphics[width=0.85\linewidth]{immagini/NET_03/vlan.png}
    \caption{VLAN=area che limita il broadcast domain}
    \label{fig:vlan}
\end{figure}


\subsection{Assegnazione e membership delle VLAN}

\subsubsection{Criteri di assegnazione}
Un dispositivo può essere assegnato a una VLAN in base a diversi criteri:

\begin{itemize}
  \item \textbf{Per porta (Port-based VLAN):} lo switch associa staticamente una VLAN a ogni porta. È il metodo più comune e lo standard definito da IEEE~802.1Q.
  \item \textbf{Per MAC address (User-based VLAN):} la VLAN è determinata dal MAC del dispositivo o dall’identità dell’utente autenticato (es. via 802.1X).
  \item \textbf{Per protocollo (Protocol-based VLAN):} introdotto da IEEE~802.1v, assegna la VLAN in base al protocollo di livello 3 (IP, IPX, ecc.).
  \item \textbf{Combinato (Cross-layer):} alcune implementazioni permettono regole gerarchiche, ad esempio prima per protocollo, poi per MAC, e infine per porta.
\end{itemize}

\subsubsection{Vista logica}
Ogni VLAN rappresenta un dominio di broadcast indipendente e, di conseguenza, è normalmente associata a una \textbf{sottorete IP dedicata}.  
La comunicazione tra VLAN diverse richiede un dispositivo L3 (router o Layer 3 switch (figura ~\ref{fig:vlan2})) che svolga la funzione di \emph{inter-VLAN routing}.

\begin{figure}[H]
    \centering
    \includegraphics[width=0.85\linewidth]{immagini/NET_03/vlan2.png}
    \caption{Physical vs logical view}
    \label{fig:vlan2}
\end{figure}

\paragraph{Router “one-armed”}  
In una configurazione one-armed router, una \textbf{singola interfaccia fisica} del router viene utilizzata per gestire \textbf{più VLAN} contemporaneamente.  
Ciò avviene tramite la creazione di \textbf{sub-interfacce virtuali}, ognuna associata a una VLAN specifica.  

\begin{figure}[H]
    \centering
    \includegraphics[width=0.65\linewidth]{immagini/NET_03/vlan3.png}
    \caption{VLAN e sottoreti IP collegate tramite sub-interfacce}
    \label{fig:vlan3}
\end{figure}

\begin{lstlisting}[language=bash,caption={Esempio di configurazione router one-armed}]
ip link add link eth0 name eth0.10 type vlan id 10
ip link add link eth0 name eth0.20 type vlan id 20
ip addr add 10.0.10.1/24 dev eth0.10
ip addr add 10.0.20.1/24 dev eth0.20
\end{lstlisting}

In questo esempio, l’interfaccia fisica \texttt{eth0} viene suddivisa in due sub-interfacce virtuali, 
\texttt{eth0.10} e \texttt{eth0.20}, rispettivamente associate alle VLAN 10 e 20.  
A ciascuna sub-interfaccia viene assegnato un indirizzo IP appartenente alla sottorete della relativa VLAN.  

In questo modo, il router può fornire servizi di \textbf{routing inter-VLAN}, agendo come \textbf{gateway predefinito} 
per ciascuna rete logica, pur utilizzando una sola porta fisica.  
Tale approccio è comune nei laboratori e negli ambienti di test, dove si vuole semplificare la topologia 
riducendo il numero di interfacce fisiche necessarie.



\subsection{Trasporto dei frame: Tagging IEEE 802.1Q}

Lo standard \textbf{IEEE 802.1Q} permette di far convivere più VLAN sullo stesso collegamento fisico.
Per farlo, inserisce all’interno dei frame Ethernet un piccolo campo aggiuntivo, chiamato \emph{tag VLAN}, 
che identifica la VLAN di appartenenza del frame.  
A seconda del tipo di collegamento, gli switch 802.1Q gestiscono questi tag in modo diverso, distinguendo tre tipologie di porte.

\subsubsection{Porte di tipo Access, Trunk e Hybrid}

\begin{itemize}
  \item \textbf{Access Port:} collega un host finale (ad esempio un PC o una stampante).  
  I frame che transitano su una porta di questo tipo sono sempre \emph{senza tag} (\emph{untagged}).  
  Lo switch associa internamente la porta a una specifica VLAN, aggiungendo o rimuovendo automaticamente il tag durante il transito interno.

  \item \textbf{Trunk Port:} collega due apparati di rete (ad esempio switch–switch o switch–router).  
  Trasporta frame appartenenti a più VLAN contemporaneamente, inviandoli e ricevendoli \emph{con tag} 802.1Q.  
  In questo modo, ciascun frame mantiene l’informazione sulla VLAN di origine anche attraverso un link condiviso.

  \item \textbf{Hybrid Port:} gestisce sia frame \emph{taggati} che \emph{non taggati}.  
  I frame non taggati vengono automaticamente associati alla \textbf{Native VLAN}, mentre quelli taggati mantengono il proprio VLAN ID.  
  Questo tipo di porta è utile quando si collegano dispositivi che supportano il tagging VLAN insieme ad altri che non lo supportano.
\end{itemize}


\subsubsection{Access links}

Un \textbf{Access link} è un collegamento che parte da una \textbf{porta di tipo Access}, 
ossia una porta configurata per appartenere a una singola VLAN.  
Questo tipo di collegamento è utilizzato per connettere dispositivi finali (come PC, stampanti o server) 
oppure piccoli hub o switch non gestiti.

I frame che transitano su una porta di tipo Access sono sempre \emph{non taggati} (\emph{untagged}):  
gli host collegati inviano e ricevono normali frame Ethernet, senza alcuna informazione sulla VLAN.  
È lo switch che, internamente, associa la porta a una VLAN e gestisce l’inserimento o la rimozione del \emph{tag 802.1Q} 
quando il frame entra o esce dalla rete VLAN-aware.  

In questo modo, i dispositivi connessi non devono essere consapevoli dell’esistenza delle VLAN:  
dal loro punto di vista fanno parte semplicemente di una rete Ethernet dedicata, tipicamente 
corrispondente a una specifica sottorete IP.

\begin{figure}[H]
    \centering
    \includegraphics[width=0.25\linewidth]{immagini/NET_03/access_link.png}
    \caption{Esempio di collegamento Access link tra host e switch VLAN-aware}
    \label{fig:access_link}
\end{figure}

\subsubsection{Access links nelle regioni legacy}

In alcuni contesti, detti \textbf{legacy regions}, un access link può estendersi attraverso 
piccole LAN composte da più switch che non supportano le VLAN (\emph{VLAN-unaware switches}).  
In questi casi, l’intera rete tradizionale viene vista dallo switch VLAN-aware come un unico segmento Ethernet 
appartenente a una sola VLAN.  

Tutti i dispositivi collegati all’interno di tale regione condividono la stessa VLAN, 
anche se gli switch intermedi non gestiscono i tag 802.1Q.  
Questo approccio consente di integrare reti esistenti non VLAN-aware in un’infrastruttura moderna basata su VLAN.

\begin{figure}[H]
    \centering
    \includegraphics[width=0.85\linewidth]{immagini/NET_03/access_link_legacy.png}
    \caption{Access link esteso in una regione legacy con switch non VLAN-aware}
    \label{fig:access_link_legacy}
\end{figure}


\subsubsection{Trunk links}

Un \textbf{Trunk link} è un collegamento che parte da una \textbf{porta di tipo Trunk}, 
utilizzato per trasportare frame appartenenti a più VLAN contemporaneamente.  
È tipicamente impiegato nei collegamenti \emph{switch–switch} o \emph{switch–router}, 
dove è necessario far transitare traffico di più reti logiche sullo stesso mezzo fisico.

A differenza delle porte Access, le porte Trunk trasmettono e ricevono \textbf{frame taggati} con l’identificatore VLAN 
secondo lo standard \textbf{IEEE 802.1Q}.  
Il tag 802.1Q consente di distinguere a quale VLAN appartiene ciascun frame, 
evitando la confusione tra traffici di reti diverse che condividono il link.

Un trunk link \textbf{non appartiene direttamente a una VLAN}, ma può trasportare:
\begin{itemize}
  \item frame provenienti da \emph{tutte} le VLAN configurate sullo switch;
  \item oppure frame appartenenti solo a un sottoinsieme di VLAN selezionate.
\end{itemize}


\begin{figure}[H]
    \centering
    \includegraphics[width=0.6\linewidth]{immagini/NET_03/trunk_link.png}
    \caption{Esempio di collegamento Trunk tra apparati VLAN-aware}
    \label{fig:trunk_link}
\end{figure}


\subsubsection{Hybrid links}

Le \textbf{Hybrid links} rappresentano un’evoluzione dei trunk link e supportano sia 
\textbf{frame taggati} sia \textbf{frame non taggati}.  
I frame taggati mantengono il proprio VLAN ID, mentre i frame non taggati vengono associati 
a una VLAN predefinita (la \textbf{Native VLAN}).

Questo tipo di collegamento è utile quando sullo stesso link devono transitare:
\begin{itemize}
  \item traffici di apparati VLAN-aware (che utilizzano frame taggati);
  \item e traffici di dispositivi legacy o VLAN-unaware (che inviano frame non taggati).
\end{itemize}

In sostanza, un hybrid link permette di far convivere traffico VLAN multiplo e traffico Ethernet standard
sullo stesso collegamento, garantendo compatibilità tra apparati di diversa generazione.  
Nelle implementazioni moderne, molti switch trattano di fatto tutti i link come \emph{ibridi}, 
in grado di gestire dinamicamente entrambe le tipologie di frame.

\begin{figure}[H]
    \centering
    \includegraphics[width=0.75\linewidth]{immagini/NET_03/hybrid_link.png}
    \caption{Esempio di Hybrid link che trasporta frame taggati e non taggati}
    \label{fig:hybrid_link}
\end{figure}
\smallskip
In figura si nota come il collegamento ibrido consenta il transito simultaneo di traffico proveniente da 
VLAN diverse, includendo anche host non VLAN-aware (VLAN~C), senza compromettere la separazione logica 
delle altre VLAN taggate.


\subsubsection{Una stazione può appartenere a più VLAN?}

In generale, un host connesso tramite una \textbf{Access port} appartiene a una sola VLAN, 
poiché i frame che transitano su quella porta sono sempre \emph{non taggati} e associati a una singola rete logica.

Tuttavia, è possibile che una stessa stazione appartenga a \textbf{più VLAN} contemporaneamente.  
Questo avviene quando la stazione dispone di una \textbf{interfaccia trunk}, in grado di inviare e ricevere 
\emph{frame taggati 802.1Q} appartenenti a VLAN diverse.

Un caso tipico è quello dei \textbf{server multi-VLAN}, che devono comunicare con più reti logiche 
(o sottoreti IP) attraverso un’unica interfaccia fisica.  
In tali scenari, il sistema operativo del server crea \textbf{sub-interfacce virtuali} 
(es. \texttt{eth0.10}, \texttt{eth0.20}), ciascuna configurata con un VLAN ID differente, 
consentendo la separazione logica del traffico pur utilizzando la stessa scheda di rete.

\begin{figure}[H]
    \centering
    \includegraphics[width=0.75\linewidth]{immagini/NET_03/multivlan_station.png}
    \caption{Esempio di stazione connessa a più VLAN tramite interfaccia trunk}
    \label{fig:multivlan_station}
\end{figure}

In sintesi, solo le stazioni dotate di interfacce \emph{VLAN-aware} possono appartenere a più VLAN: 
sono esse a gestire il tagging e l’instradamento del traffico tra le diverse reti logiche.


\newpage
\maketitle
{\LARGE \textbf{Laboratorio}} \\[0.5em]
\subsection{Laboratorio 3: Configurazione VLAN e router one-armed}

In questo laboratorio si analizza il funzionamento delle VLAN e del \textbf{routing inter-VLAN} 
attraverso un router connesso tramite una singola interfaccia fisica (\emph{one-armed router}).  
L’obiettivo è comprendere come i frame vengano trasportati attraverso collegamenti di tipo 
\emph{Access}, \emph{Trunk} e \emph{Hybrid} secondo lo standard IEEE 802.1Q.

\subsection*{Topologia di rete}
La rete è composta da:
\begin{itemize}
  \item uno \textbf{switch VLAN-aware} con tre porte configurate come Access (VLAN 10 e 20) 
        e una porta configurata come Trunk verso il router;
  \item due host collegati alle porte Access, ciascuno appartenente a una VLAN distinta;
  \item un router configurato con sub-interfacce virtuali (\texttt{eth0.10}, \texttt{eth0.20}) 
        per fornire connettività inter-VLAN.
\end{itemize}

\begin{figure}[H]
    \centering
    \includegraphics[width=0.75\linewidth]{immagini/NET_03/lab3_topology.png}
    \caption{Topologia del Laboratorio 3: VLAN e router one-armed}
    \label{fig:lab3_topology}
\end{figure}

\subsection*{Configurazione di esempio}
\begin{lstlisting}[language=bash,caption={Configurazione delle VLAN sul router one-armed}]
# Creazione delle sub-interfacce VLAN
ip link add link eth0 name eth0.10 type vlan id 10
ip link add link eth0 name eth0.20 type vlan id 20

# Assegnazione degli indirizzi IP (gateway delle rispettive VLAN)
ip addr add 10.0.10.1/24 dev eth0.10
ip addr add 10.0.20.1/24 dev eth0.20

# Attivazione delle interfacce
ip link set eth0.10 up
ip link set eth0.20 up
\end{lstlisting}

Sul lato switch:
\begin{lstlisting}[language=bash,caption={Configurazione delle VLAN sullo switch}]
# Creazione VLAN
vlan 10
vlan 20

# Assegnazione delle porte
interface eth1
 switchport mode access
 switchport access vlan 10

interface eth2
 switchport mode access
 switchport access vlan 20

# Porta trunk verso il router
interface eth0
 switchport mode trunk
 switchport trunk allowed vlan 10,20
\end{lstlisting}
\subsection*{Funzionamento del laboratorio}

Il laboratorio ha lo scopo di mostrare in modo pratico il funzionamento delle \textbf{VLAN} 
e del \textbf{router one-armed}, ovvero una configurazione in cui un unico collegamento fisico 
tra router e switch trasporta il traffico di più VLAN tramite \textbf{frame taggati IEEE~802.1Q}.

\paragraph{Isolamento del traffico}
Gli host collegati alle \textbf{porte Access} appartengono ciascuno a una VLAN distinta.  
I frame che transitano su queste porte sono \emph{non taggati} e vengono separati dallo switch in base alla VLAN di appartenenza.  
In questo modo, gli host di VLAN diverse non possono comunicare direttamente: lo switch isola i domini di broadcast 
e impedisce la comunicazione a livello 2.

\paragraph{Routing inter-VLAN}
Per permettere la comunicazione tra VLAN diverse, il traffico deve passare attraverso il router.  
La connessione tra router e switch avviene tramite una \textbf{porta di tipo Trunk}, che trasporta i frame di più VLAN 
aggiungendo un tag 802.1Q a ciascun frame.  
Sul router, l'interfaccia fisica (ad esempio \texttt{eth0}) è suddivisa in più \textbf{sub-interfacce virtuali} 
(\texttt{eth0.10}, \texttt{eth0.20}, ecc.), ognuna configurata con:
\begin{itemize}
  \item un \textbf{VLAN ID} specifico;
  \item un indirizzo IP che funge da \textbf{gateway} per la relativa VLAN.
\end{itemize}

Quando un host della VLAN~10 invia un pacchetto verso un host della VLAN~20:
\begin{enumerate}
  \item il frame raggiunge lo switch sulla porta Access e viene inoltrato sul trunk verso il router con tag VLAN~10;
  \item il router riceve il frame su \texttt{eth0.10}, lo elabora a livello 3 e decide di inoltrarlo sulla sub-interfaccia \texttt{eth0.20};
  \item il router rimanda il frame allo switch, questa volta con tag VLAN~20;
  \item lo switch rimuove il tag e lo invia alla porta Access corrispondente alla VLAN~20.
\end{enumerate}

\begin{figure}[H]
    \centering
    \includegraphics[width=0.85\linewidth]{immagini/NET_03/example.png}
    \caption{Comunicazione tra diverse vlan}
    \label{fig:example}
\end{figure}


\paragraph{Verifica del comportamento}
Nel laboratorio si può verificare che:
\begin{itemize}
  \item gli host della stessa VLAN comunicano direttamente a livello~2, senza passare dal router;
  \item la comunicazione tra VLAN diverse avviene tramite il router (routing inter-VLAN);
  \item tutto il traffico tra router e switch è trasportato sul link trunk mediante frame taggati 802.1Q.
\end{itemize}

\paragraph{Osservazione pratica}
Catturando il traffico con \texttt{tcpdump} o \texttt{Wireshark} sull’interfaccia trunk, 
è possibile osservare i \textbf{tag VLAN} all’interno dei frame Ethernet.  
Ciò consente di verificare visivamente il meccanismo di separazione e instradamento del traffico 
tra le diverse VLAN.




\newpage
\subsection{Sicurezza delle VLAN e vulnerabilità di livello 2}

Le VLAN migliorano l’isolamento logico del traffico, ma non eliminano le minacce 
presenti a livello di collegamento.  
Un attaccante connesso alla rete locale può sfruttare debolezze dei protocolli 
di livello~2 (Ethernet e ARP) o configurazioni errate degli switch per intercettare,
modificare o dirottare il traffico di rete.

\subsubsection{Minacce principali}

\paragraph{1) MAC Flooding (CAM Overflow)}
Gli switch mantengono in memoria una \textbf{Content Addressable Memory (CAM)}, 
che associa indirizzi MAC a porte fisiche.  
Un attaccante può inviare migliaia di frame con indirizzi MAC falsi, 
riempiendo la tabella CAM e provocando un \emph{overflow}.  
Quando la tabella è satura, lo switch non riesce più a determinare su quale porta si trova un determinato MAC 
e inizia a inoltrare i frame in \textbf{broadcast}, esponendo il traffico all’attaccante.

\begin{figure}[H]
    \centering
    \includegraphics[width=0.75\linewidth]{immagini/NET_03/mac_attack.png}
    \caption{Esempio di MAC flooding}
    \label{fig:mac_attack}
\end{figure}

\emph{Mitigazione:} abilitare la \textbf{port security}, limitando il numero di MAC address appresi per ciascuna porta, 
e disattivare l’apprendimento dinamico dove non necessario.

\paragraph{2) ARP Spoofing / Poisoning}
Il protocollo ARP non prevede autenticazione, quindi un attaccante può inviare \textbf{risposte ARP falsificate} 
per associare il proprio MAC all’indirizzo IP del gateway o di altri host nella stessa VLAN.  
In questo modo, intercetta o altera il traffico tra due dispositivi (\emph{Man-in-the-Middle}).


\emph{Mitigazione:} configurare \textbf{ARP statici} per i dispositivi critici, utilizzare strumenti di monitoraggio 
come \texttt{arpwatch} o implementare sistemi di protezione come \texttt{Dynamic ARP Inspection (DAI)} sugli switch gestiti.

\paragraph{3) VLAN Hopping}
Questa categoria di attacchi consente a un host di inviare o ricevere traffico appartenente a una VLAN diversa 
da quella assegnata, violando l’isolamento logico.  
Le due tecniche principali sono:
\begin{itemize}
  \item \textbf{Basic VLAN hopping:} l’attaccante sfrutta il protocollo DTP (\emph{Dynamic Trunking Protocol}) 
        per negoziare automaticamente una connessione di tipo trunk con lo switch, ottenendo accesso a più VLAN.
    \begin{figure}[H]
        \centering
        \includegraphics[width=0.5\linewidth]{immagini/NET_03/basic_vlan_hopping.png}
        \caption{Collegamento tra due switch tramite porte trunk che trasportano il traffico di più VLAN sullo stesso link fisico.}
        \label{fig:vlan_hopper}
    \end{figure}
  \item \textbf{Double Tagging:} il frame viene costruito con due tag 802.1Q annidati.  
        Il primo tag (relativo alla VLAN nativa) viene rimosso dallo switch di ingresso, 
        lasciando il secondo, che identifica la VLAN bersaglio.
    \begin{figure}[H]
        \centering
        \includegraphics[width=0.5\linewidth]{immagini/NET_03/double_tagging.png}
        \caption{Esempio di attacco \textbf{VLAN Double Tagging}: un host malevolo appartenente alla VLAN~10 inserisce due tag 802.1Q (VLAN~10 e VLAN~40). 
    Il primo switch rimuove il tag della VLAN~nativa (10) e inoltra il frame, che conserva il secondo tag (40). 
    Il frame attraversa quindi il trunk ed entra nella VLAN~40, violando l’isolamento tra VLAN.}
        \label{fig:double_}
    \end{figure}
\end{itemize}

\emph{Mitigazione:} disabilitare DTP e configurare manualmente le porte come \textbf{Access} dove necessario;  
evitare l’uso della \textbf{VLAN~1} come VLAN nativa e assegnare VLAN native dedicate non utilizzate per il traffico utente.
\paragraph{4) Attacchi ai protocolli di controllo}

Oltre agli attacchi diretti alle VLAN, anche i protocolli di \textbf{controllo e gestione} 
utilizzati dagli switch di livello~2 possono essere sfruttati da un attaccante per alterare 
la topologia della rete o raccogliere informazioni sensibili.  
Tra i principali:


\subparagraph{Spanning Tree Attack (BPDU spoofing)}
\begin{itemize}
  \item \textbf{Cosa fa lo STP:} gli switch si scambiano BPDU (bridge protocol data units) per eleggere il \emph{root bridge} e stabilire quali porte siano forwarding o blocking, evitando loop.
  \item \textbf{Come attacca l'avversario:} l'attaccante invia BPDU falsi con una priorità molto bassa (o con un bridge ID fittizio), facendo credere agli switch che il suo dispositivo sia il nuovo root.
  \item \textbf{Effetto pratico:} la topologia si ricalcola; alcuni link possono essere forzati in forwarding creando percorsi non previsti, perdita di connettività o instradamento del traffico attraverso il nodo dell'attaccante.
  \item \textbf{Mitigazione:} \emph{BPDU Guard} spegne la porta se riceve BPDU su una porta che dovrebbe essere una porta access (cioè verso host), mentre \emph{Root Guard} impedisce a un certo segmento di diventare root forzando il comportamento previsto.
\end{itemize}

\subparagraph{VTP Attack (VLAN Trunking Protocol)}
\begin{itemize}
  \item \textbf{Cosa fa VTP:} permette di distribuire automaticamente la lista delle VLAN a tutti gli switch del dominio VTP.
  \item \textbf{Come attacca l'avversario:} un dispositivo malevolo si presenta come \texttt{server VTP} con una \emph{revision number} superiore; gli altri switch accettano la nuova configurazione e sovrascrivono le VLAN locali.
  \item \textbf{Effetto pratico:} le VLAN possono essere cancellate o modificate su larga scala, causando indisponibilità o perdita dell'isolamento tra reti.
  \item \textbf{Mitigazione:} impostando VTP in \emph{transparent} o disabilitandolo si evita che uno switch accetti e propaghi automaticamente configurazioni provenienti da fonti non affidabili.
\end{itemize}

\subparagraph{Cisco Discovery Protocol Attack (information leakage)}
\begin{itemize}
  \item \textbf{Cosa fa CDP:} i dispositivi Cisco pubblicano informazioni (IP, modelli, versioni) su CDP verso i vicini.
  \item \textbf{Come attacca l'avversario:} un host malintenzionato cattura i pacchetti CDP o ascolta il traffico e ottiene dettagli utili per attacchi mirati (es. versioni vulnerabili).
  \item \textbf{Effetto pratico:} ricognizione facilitata: l'attaccante conosce quali dispositivi e software colpire.
  \item \textbf{Mitigazione:} disabilitando CDP sulle porte utenti si riduce la quantità di informazioni esposte ai terminali non fidati.
\end{itemize}






%%%%%%%%%%%%%%%%%       laboratorio         %%%%%%%%%%
\newpage
\maketitle
{\LARGE \textbf{Laboratorio}} \\[0.5em]
\subsection{Laboratorio 4: VLAN Hopping e Double Tagging}

\subsubsection{Scenario}
L’obiettivo è dimostrare un attacco di \textbf{double tagging 802.1Q}: un host nella \textbf{VLAN nativa (VLAN~1)} tenta di inviare traffico a una vittima in \textbf{VLAN~20} \emph{senza} routing inter-VLAN, sfruttando la rimozione del tag della VLAN nativa sul primo switch.

\begin{figure}[H]
  \centering
  \includegraphics[width=.8\linewidth]{immagini/NET_03/double_tagging.png}
  \caption{Attacco Double Tagging su trunk con VLAN nativa impostata a 1.}
  \label{fig:double_tagging}
\end{figure}

\subsubsection{Topologia del laboratorio}
La topologia è in \autoref{fig:lab4_topology}: attaccante in \textbf{VLAN 1 (nativa)} connesso a una porta che insiste sul \emph{trunk}; vittima in \textbf{VLAN 20} su uno switch a valle; nessun router tra le VLAN.

\begin{figure}[H]
  \centering
  \includegraphics[width=.75\linewidth]{immagini/NET_03/lab4_topology.png}
  \caption{Lab 4 — Topologia: attaccante su VLAN~1 (nativa) verso trunk; vittima in VLAN~20; \emph{no} inter-VLAN routing.}
  \label{fig:lab4_topology}
\end{figure}

\noindent\textit{Nota operativa.} Le porte \texttt{eth3} dei due switch formano il \emph{trunk} con \textbf{VLAN nativa = 1}. L’attaccante è su una porta \emph{access} in VLAN~1; gli host \texttt{client-3} (VLAN~10) e \texttt{client-4} (VLAN~20) sono su porte \emph{access} degli switch opposti.

\subsubsection{Prerequisiti perché l’attacco riesca}
\begin{itemize}
  \item esiste un \textbf{link trunk} tra due switch;
  \item \textbf{stessa VLAN nativa} configurata su entrambe le estremità del trunk (qui: VLAN~1);
  \item l’attaccante è connesso a \textbf{VLAN~1} (nativa) e può inviare frame con \emph{doppio tag};
  \item la vittima appartiene a una \textbf{VLAN diversa} (qui: VLAN~20).
\end{itemize}

\subsubsection{Funzionamento dell’attacco (passo–passo)}
L’attaccante costruisce un frame con \textbf{due tag 802.1Q annidati}:
\[
\underbrace{\texttt{Tag esterno} = \text{VLAN nativa (1)}}_{\text{verrà rimosso dal primo switch}} 
\quad\big|\quad 
\underbrace{\texttt{Tag interno} = \text{VLAN bersaglio (20)}}_{\text{rimarrà visibile al secondo switch}}.
\]

\begin{enumerate}
  \item \textbf{Invio dal nodo malevolo.} Il frame parte dall’attaccante con due tag: \texttt{VLAN 1} (esterno) e \texttt{VLAN 20} (interno).
  \item \textbf{Primo switch (ingresso trunk).} Per definizione, i frame \emph{taggati con la VLAN nativa} sul trunk vengono trattati come \emph{untagged}: lo switch \textbf{rimuove il tag esterno} (VLAN~1) e inoltra il frame sul trunk \emph{senza} quel tag.
  \item \textbf{Attraversamento del trunk.} Il frame prosegue e, non avendo più il tag esterno, \textbf{rimane solo il tag interno} (VLAN~20) nel payload.
  \item \textbf{Secondo switch (uscita trunk).} Vede un frame \emph{ancora taggato} con \textbf{VLAN~20} (il tag interno) e lo inoltra correttamente all’\emph{access port} della VLAN~20.
  \item \textbf{Consegna alla vittima.} L’host in VLAN~20 riceve il frame come se provenisse da un nodo legittimo della propria VLAN.
\end{enumerate}

\noindent\textbf{Unidirezionalità.} La risposta della vittima \emph{non} segue lo stesso percorso: l’host in VLAN~20 non inserisce doppio tag e il frame di risposta non potrà rientrare verso l’attaccante in VLAN~1 senza routing. Per questo l’attacco è tipicamente \textbf{mono\-direzionale} (utile per inviare pacchetti/sondare servizi, meno per instaurare dialoghi completi).

\subsubsection{Esecuzione pratica (Linux)}
\begin{lstlisting}[language=bash,caption={Costruzione di un frame con doppio tag via sub-interfacce annidate}]
ip link add link eth0 name eth0.1 type vlan id 1
ip link set eth0.1 up
ip link add link eth0.1 name eth0.1.20 type vlan id 20
ip link set eth0.1.20 up
ip addr add 10.0.20.250/24 dev eth0.1.20
arp -s 10.0.20.102 <MAC-vittima> -i eth0.1.20
ping 10.0.20.102
\end{lstlisting}

\noindent In questo modo lo \texttt{stack} inserisce \textbf{due tag 802.1Q} (outer=1, inner=20). Il \texttt{ping} raggiunge la vittima, ma la risposta generalmente non torna all’attaccante (vedi unidirezionalità).

\paragraph{Cosa osservare con lo sniffer}
\begin{itemize}
  \item sul \textbf{primo switch lato trunk}: si vedono frame che escono \emph{senza} tag nativo e \emph{con} il tag interno (20) ancora presente;
  \item sul \textbf{secondo switch}: si vedono frame \textbf{con VLAN~20} che vengono consegnati alla porta \emph{access} della vittima;
  \item sull’host attaccante: assenza di risposte ICMP (o ARP) dalla vittima per la natura monodirezionale.
\end{itemize}

\paragraph{Limitazioni pratiche}
\begin{itemize}
  \item se la \textbf{VLAN nativa non è usata} (o è diversa e dedicata), il meccanismo di rimozione del primo tag non produce l’effetto desiderato;
  \item molti switch moderni applicano \textbf{filtri} su frame anomali (ad es.\ doppio tag con \emph{outer} uguale alla nativa).
\end{itemize}

\paragraph{Mitigazioni}
\begin{itemize}
  \item \textbf{Disabilitare l’auto-trunking (DTP)} e configurare \emph{manual\-mente} le porte utente come \texttt{access};
  \item \textbf{Non usare VLAN~1 come nativa}; scegliere una \textbf{VLAN nativa dedicata} e non operativa per il traffico utente;
  \item \textbf{Consentire sul trunk solo le VLAN necessarie} (\emph{allowed VLAN list});
  \item abilitare controlli di \textbf{storm/unknown-unicast} e ispezioni su frame con \emph{double tag} dove supportato;
  \item monitoraggio con IDS/port mirroring su link trunk per \textbf{pattern di doppio tagging}.
\end{itemize}


\subsection*{Conclusioni}
Il \textbf{double tagging} sfrutta la gestione della \textbf{VLAN nativa} sui trunk per far accettare a uno switch un \emph{tag interno} verso una VLAN diversa, \emph{senza} routing. È spesso \textbf{monodirezionale}. La difesa passa da: hardening dei trunk (nativa dedicata, allowed-VLAN), porte utente in \texttt{access} statico, autenticazione 802.1X e controllo a livello di firewall/IDS.
