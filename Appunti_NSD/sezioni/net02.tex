\newpage
\maketitle
\section{NET\_02}

\subsection{Sicurezza delle Reti Ethernet LAN (Livello 2)}

\subsection{Perché la LAN Ethernet è fragile per natura}
Ethernet nasce per \emph{autoconfigurarsi}: gli switch imparano indirizzi e percorsi (MAC learning), i protocolli di controllo (STP, ARP, DHCP) si basano su broadcast e sulla mancanza di autenticazione a L2. Questo rende immediati tre vettori: \textbf{osservare} (eavesdropping), \textbf{manipolare} (spoofing/MITM) e \textbf{interrompere} (DoS). Le minacce si organizzano in quattro famiglie: (i) accesso a rete/sistemi, (ii) confidenzialità, (iii) disponibilità, (iv) integrità.

\subsubsection{Frame, indirizzamento e forwarding}
Gli indirizzi Mac sono a 48 bit.\\
nella versione originale dell ethernet la tipologia del frame veniva usata per il demultiplexing del layer superiore per esemio 0x800=IP; mentre in 802.3 indica la lunghezza oppure il tipo in particolare se il frame supera i $0x0600$ ($1536_{10}$) allora indica la tipologia di frame altrimenti LLC per il demultiplexing e indica il payload size. Se frame inferiore ai 46 bit allora viene utilizzato del padding.\\
Il primo bit dell’indirizzo MAC indica se l’indirizzo è unicast (0) o di gruppo/multicast (1); il secondo bit distingue tra indirizzi globali (0, assegnati dal produttore) e locali (1, configurabili via software o driver).

\paragraph{Multiport Repeaters (Hub)}
Gli hub, o ripetitori multiporta, operano come un bus condiviso: tutto il traffico ricevuto viene rigenerato su tutte le porte. 
Appartengono quindi a un unico \textbf{dominio di collisione} e non offrono isolamento tra host; per questo sono oggi sostituiti dagli switch.\\

\paragraph{Bridge/Switches}:\\
Gli switch possono operare in:
\begin{itemize}
  \item \textbf{Store\&Forward}: lettura completa del frame (memorizzazione su buffer), controllo CRC, scarto dei frame \emph{runt}(<64 bytes too short)/troppo lunghi oppure se fallisce il CRC; look up nella tabella e forwarding.
  \item \textbf{Cut-through}: lettura solo fino all'indirizzo, nessun check di integrità, look-up, forwarding.
\end{itemize}
Il \textbf{Forwarding Database} (FDB) mappa \texttt{MAC→porta}; le entry dinamiche sono apprese (MAC learning) e scadono con \emph{ageing} tipicamente nell’ordine dei 300\,s; altrimenti impostate staticamente da un sysadmin o da un db. Se la destinazione è sconosciuta, lo switch effettua \emph{flooding}.
\paragraph{Address Learning}
Un frame arriva alla porta X quindi deve provenire dalla LAN connessa dalla porta X, il source address viene usato per l update del forwarding DB. Se arriva un frame da un source addr non presente nella tabella allora viene creata la entry con $age=0$; se invece arriva un entry già presente viene refreshata la age di quella entry.

\begin{figure}[H]
    \centering
    \includegraphics[width=.85\linewidth]{immagini/NET_02/adress_learning_1.png}
    \caption{Topologia rete}
    \label{fig:lab_1}
\end{figure}
Infine se arriva un frame da un source addr già presente nella tabella ma sotto una porta differente allora viene aggiornata la entry e refresh della age.

\begin{figure}[H]
    \centering
    \includegraphics[width=.85\linewidth]{immagini/NET_02/address_learning_2.png}
    \caption{Topologia rete}
    \label{fig:lab_2}
\end{figure}


\subsubsection{Topologie e controllo dei loop: STP}
Quando in una rete Ethernet esistono collegamenti ridondanti, possono formarsi \emph{loop} a \textbf{Livello 2 (L2, Data Link)}: i frame possono circolare all’infinito tra switch, saturando la rete. 
Per evitare questo, entra in gioco lo \textbf{Spanning Tree Protocol}. 
L’idea è eleggere uno \emph{switch radice} (\textbf{root bridge}) e disattivare alcune porte in modo che la topologia effettiva sia un albero evitando così loop, pur lasciando i link ridondanti pronti a subentrare in caso di guasti.

Gli switch si mettono d’accordo scambiandosi messaggi di controllo chiamati \textbf{BPDU (Bridge Protocol Data Unit)}. 
Ogni BPDU contiene l’identità dello switch (\emph{Bridge ID}, che include \emph{priorità} e MAC) e i \emph{costi} dei percorsi. 
Viene eletto un \emph{root bridge}; e poi per ogni altro switch, \textbf{STP} sceglie:
\begin{itemize}
  \item una \textbf{porta radice (root port)}: la porta verso il root bridge con costo minore;
  \item eventuali porte in eccesso vengono messe in stato \textbf{bloccato (blocking)} per rompere i cicli.
\end{itemize}
Il risultato è che solo alcune porte sono di forwarding, mentre le altre restano \textbf{bloccate}; se un link o uno switch si guastano allora STP riconfigura la topologia evitando nuovamente loop.

Sul piano della sicurezza, però, STP ha un limite: a L2 non c’è \textbf{autenticazione} dei messaggi di controllo. 
Un host malevolo può inviare \textbf{BPDU} artefatte e farsi eleggere \emph{root bridge} (alzando la priorità o manipolando i costi), dirottando o interrompendo il traffico. 
Per questo, in produzione si usano contromisure come \emph{BPDU Guard}, \emph{Root Guard}, \emph{portfast} solo sugli host, e piani di controllo isolati.


\subsubsection{Adattamento del Livello 3 su Livello 2: DHCP, ARP e NDP}

L’interoperabilità tra il \textbf{Livello 2 (Data Link)} e il \textbf{Livello 3 (Network)} è assicurata da una serie di protocolli di adattamento che consentono agli host di configurare automaticamente i propri parametri IP e di risolvere gli indirizzi fisici (MAC) dei dispositivi vicini. 
Questi protocolli, fondamentali per il funzionamento delle reti IP, nascono però in un contesto di fiducia implicita e \textbf{assenza di autenticazione}, diventando quindi bersagli ideali per attacchi di spoofing e manipolazione del traffico.

\paragraph{DHCP — Dynamic Host Configuration Protocol (IPv4).}
Il \textbf{DHCP} automatizza l’assegnazione degli indirizzi IP e dei parametri di rete,. 
Opera in modalità \emph{client–server} e sfrutta il meccanismo di broadcasting a livello Ethernet per raggiungere il server anche quando il client non ha ancora un IP. 

Il protocollo segue un tipico \textbf{handshake a quattro fasi}:
\begin{enumerate}
  \item \textbf{DHCP Discover} — il client trasmette in broadcast (\texttt{255.255.255.255}) per cercare un server disponibile;
  \item \textbf{DHCP Offer} — il server risponde offrendo un indirizzo IP e altri parametri (gateway, DNS, lease time);
  \item \textbf{DHCP Request} — il client accetta esplicitamente un’offerta specifica;
  \item \textbf{DHCP ACK} — il server conferma l’assegnazione e crea una voce di \emph{lease} nel proprio database.
\end{enumerate}
Ogni lease è temporaneo e può essere rinnovato tramite messaggi \emph{DHCP Renew/Rebind}.  
Quando un host si trova in una rete diversa dal server, la comunicazione avviene tramite un \textbf{DHCP relay agent} — spesso un router — che incapsula i messaggi Discover/Request e li inoltra verso il server remoto, mantenendo così la visibilità dell’origine (campo \texttt{giaddr}).  
\begin{figure}[H]
    \centering
    \includegraphics[width=0.5\linewidth]{immagini//NET_02/DHCP.png}
    \label{fig:placeholder1}
\end{figure}

\textbf{Sicurezza:} DHCP non autentica né il client né il server. Un host malevolo può rispondere più velocemente del server legittimo (\textbf{DHCP spoofing}) e assegnare gateway o DNS controllati, dirottando il traffico.  
Per mitigare questi scenari, gli switch moderni implementano \textbf{DHCP Snooping}: una funzione che registra le associazioni IP–MAC–porta–VLAN apprese dai messaggi DHCP legittimi, utile anche per alimentare altri controlli di sicurezza come la \emph{Dynamic ARP Inspection}.

\paragraph{ARP — Address Resolution Protocol (IPv4).}
L'\textbf{ARP} consente di scoprire l’indirizzo MAC associato a un indirizzo IP all’interno della stessa LAN.  
Quando un host deve inviare un pacchetto IP verso una destinazione della propria subnet, interroga la rete inviando un messaggio \textbf{ARP Request} in broadcast contenente l’indirizzo IP cercato.  
L’host corrispondente risponde con un \textbf{ARP Reply} unicast, fornendo il proprio MAC address.  
Ogni sistema mantiene una \textbf{cache ARP} che memorizza temporaneamente queste associazioni per ridurre il numero di richieste future (tipicamente 20 minuti su sistemi UNIX-like).
\begin{figure}[H]
    \centering
    \includegraphics[width=0.5\linewidth]{immagini//NET_02/ARP.png}
    \caption{}
    \label{fig:placeholder2}
\end{figure}

\textbf{Debolezze:} ARP è \emph{stateless} e non prevede autenticazione.  
Un attaccante può quindi inviare \textbf{ARP Reply falsificati} (anche senza richiesta) per associare un IP legittimo al proprio MAC address: è il classico \textbf{ARP poisoning}, che consente di intercettare, modificare o bloccare il traffico (attacco \emph{Man-in-the-Middle}).  
Meccanismi come \textbf{Dynamic ARP Inspection (DAI)}, basati sulle tabelle di DHCP Snooping, permettono di bloccare risposte ARP incoerenti rispetto alle associazioni IP–MAC note.

\paragraph{NDP — Neighbor Discovery Protocol (IPv6).}
Nel mondo IPv6, il \textbf{Neighbor Discovery Protocol (NDP)}, sostituisce ARP e parte del ruolo di DHCP.  
Basato su \textbf{ICMPv6}
A differenza di ARP, NDP usa \textbf{multicast} anziché broadcast, riducendo l’impatto sulla rete e migliorando la scalabilità.  
Tuttavia, condivide la stessa assenza di autenticazione: un host può fingere di essere un router o un vicino legittimo, portando a \emph{neighbor spoofing} o \emph{router advertisement flooding}.
\subsubsection{Vulnerabilità e minacce principali}

La sicurezza di una rete Ethernet dipende fortemente dall’affidabilità del Livello 2, ma i protocolli su cui si basa — come ARP, DHCP e STP — sono stati progettati per un ambiente fidato, senza autenticazione o cifratura. 
Questo rende l’intera architettura LAN intrinsecamente vulnerabile ad attacchi che puntano al controllo dell’accesso fisico, alla manipolazione del traffico e al degrado delle prestazioni. 
Le principali minacce si raggruppano in quattro categorie: \textbf{(1) accesso alla rete}, \textbf{(2) riservatezza del traffico}, \textbf{(3) integrità e manipolazione}, \textbf{(4) disponibilità e prestazioni}.

\paragraph{Accesso alla rete}.\\ 

\textbf{Accesso fisico non autorizzato.}  
L’attacco più basilare consiste nel collegarsi fisicamente a una porta Ethernet lasciata attiva e non monitorata. 
In ambienti non presidiati (aule, uffici, open space), un attaccante può semplicemente inserire il proprio dispositivo o un piccolo switch/access point (\emph{rogue device}), espandendo la rete interna senza autorizzazione.  
Poiché lo standard Ethernet non prevede autenticazione a livello di porta, la connessione risulta immediatamente operativa.  
Questo tipo di “\emph{join} fisico” rappresenta il punto d’ingresso di molte compromissioni LAN.

\textbf{Accesso remoto e ricognizione.}  
Una volta ottenuto l’accesso (fisico o logico), l’attaccante può mappare la topologia di rete sfruttando protocolli di base:
\begin{itemize}
  \item \emph{ARP scanning} — per enumerare gli indirizzi IP attivi nella LAN;
  \item richieste \emph{DHCP} — per dedurre l’intervallo di indirizzi disponibili e i parametri di rete (gateway, DNS);
  \item \emph{port scanning} — per identificare i servizi esposti e i sistemi operativi in uso.
\end{itemize}
Queste informazioni consentono di costruire una mappa logica della rete e di selezionare successivi obiettivi di attacco (\emph{target profiling}).

\paragraph{Compromissione e controllo dello switch.}
Gli switch possono diventare obiettivi diretti se il \emph{management plane} è esposto o debolmente protetto. Molti dispositivi arrivano con
\textbf{credenziali di default} (o addirittura senza password) e spesso prevedono un \textbf{reset fisico} che ripristina i parametri di fabbrica:
in entrambi i casi un attaccante può ottenere l’accesso amministrativo. Una volta dentro, è possibile \textbf{dirottare il traffico} abbassando link
strategici, \textbf{farsi eleggere root bridge} in \textit{Spanning Tree} aumentando la priorità dello switch, o \textbf{indurre DoS} su link selezionati. 
Inoltre, a seconda dei \emph{protocolli di gestione} abilitati, l’attaccante può \textbf{attivare il port mirroring} per intercettare i flussi e 
(\emph{in alcuni scenari}) ottenere visibilità o accesso a VLAN non previste.

\noindent
Dal lato protocolli L2, l’assenza di autenticazione a livello Ethernet rende possibile \textbf{manipolare STP} (fingendosi switch legittimo) 
per alterare la topologia o causare riconvergenze continue: è la base di diversi vettori di \emph{denial-of-service}.

\paragraph{Riservatezza e intercettazione.}

\textbf{Eavesdropping (intercettazione).}  
In una rete Ethernet tradizionale basata su hub, tutto il traffico viene propagato su tutte le porte, rendendo facile l’intercettazione passiva (\emph{sniffing}).  
Anche negli switch moderni, l’attacco resta possibile tramite:
\begin{itemize}
  \item installazione fisica di un dispositivo di ascolto (\emph{tap}) su un cavo in rame o fibra ottica;
  \item abilitazione di una scheda di rete in \emph{promiscuous mode} per ricevere frame non destinati al proprio MAC (annulla filtering MAC);
  \item abuso della funzione di \emph{port mirroring} sugli switch, utile per intercettare il traffico di altre porte.
\end{itemize}

\textbf{MAC flooding.}  
Gli switch memorizzano le associazioni \texttt{MAC→porta} nella CAM (Content Addressable Memory) o FDB (Forwarding Database).  
Un attaccante può saturare questa memoria inviando migliaia di frame con indirizzi MAC sorgente casuali.  
Quando la tabella si riempie, lo switch passa alla modalità \emph{flooding}, inoltrando i frame su tutte le porte come un hub.  
Questo permette di catturare traffico unicast normalmente privato e, in certi casi, di rompere l’isolamento tra VLAN (\emph{cross-VLAN leakage}).

\textbf{MAC spoofing.}  
Manipolando l’indirizzo MAC sorgente dei frame inviati, un attaccante può sostituirsi a un host legittimo già presente nella FDB.  
Il risultato è un \emph{hijacking} del traffico diretto alla vittima, che può essere intercettato, modificato o semplicemente bloccato.  
Lo spoofing è efficace perché Ethernet non verifica la coerenza tra il MAC dichiarato nel frame e quello della scheda di rete.

\paragraph{Integrità del traffico e attacchi Man-in-the-Middle (MITM)} .\\

\textbf{ARP poisoning.}  
Approfittando del fatto che l’ARP accetta qualunque risposta non autenticata, e anche quelle non richieste, un attaccante può inviare \emph{ARP replies} falsificati per associare il proprio MAC address all’indirizzo IP di un altro nodo (ad esempio il gateway).  
In questo modo intercetta tutto il traffico tra la vittima e il router, realizzando un classico attacco \emph{Man-in-the-Middle (MITM)}.  
Varianti di questo attacco esistono anche in IPv6 sotto forma di \emph{Neighbor Advertisement spoofing} contro NDP.

\textbf{DHCP poisoning.}  
Simile nel principio, l’attacco DHCP poisoning consiste nel rispondere alle richieste DHCP più rapidamente del server legittimo.  
Il client riceve così parametri falsi (indirizzo IP, gateway, DNS), che permettono all’attaccante di deviare il traffico verso sistemi controllati o intercettarlo.

\textbf{Session hijacking.}  
Una volta intercettato il traffico (via ARP o DHCP poisoning), è possibile analizzare le sessioni di livello superiore (ad esempio TCP) e riprodurle, utilizzando numeri di sequenza o cookie d’autenticazione per impersonare un utente o un servizio.

\textbf{Replay attack.}  
Consiste nel riutilizzare pacchetti validi intercettati in precedenza — ad esempio messaggi di controllo o di autenticazione — per ottenere accesso o indurre comportamenti anomali nei dispositivi di rete.  
In mancanza di firme digitali o timestamp, questi messaggi vengono accettati come legittimi.

\paragraph{Disponibilità e attacchi di Denial of Service (DoS)}.\\

\textbf{STP DoS e manipolazione topologica}  
Il protocollo \textbf{Spanning Tree Protocol} non prevede autenticazione dei messaggi di controllo (\textbf{BPDU, Bridge Protocol Data Units}).  
Un attaccante può sfruttare questa debolezza per:
\begin{itemize}
  \item eleggersi come \emph{root bridge} forzando il traffico a passare attraverso il proprio nodo;
  \item inondare la rete di BPDU falsi, causando continui ricalcoli dell’albero di spanning e interruzioni periodiche dei collegamenti (\emph{STP reconvergence loops}).
\end{itemize}

\begin{figure}[H]
    \centering
    \includegraphics[width=0.5\linewidth]{immagini//NET_02/MITM.png}
    \caption{}
    \label{fig:placeholder}
\end{figure}

\textbf{Resource exhaustion e flooding.}  
Un altro vettore comune consiste nel sovraccaricare il piano di controllo degli switch o dei router.  
Attacchi di \emph{unknown-unicast flooding} e tempeste di broadcast (\emph{broadcast storms}) possono saturare la banda o la memoria, impedendo il corretto forwarding dei frame.  
Simili risultati possono essere ottenuti generando una quantità eccessiva di richieste ARP o DHCP, portando al collasso del servizio (\emph{DoS a livello L2}).

\subsubsection{Contromisure: dal minimo sindacale al robusto}

Per affrontare le vulnerabilità strutturali di Ethernet e le minacce che ne derivano, è fondamentale adottare un approccio stratificato. Le contromisure spaziano da soluzioni basilari, come la segmentazione della rete, a soluzioni avanzate che includono l'uso di crittografia L2 e l'autenticazione a livello di porta.

\paragraph{Router-based security (segmentazione L3).}
Sostituire (o affiancare) lo switch centrale con un \textbf{router IP} spezza la LAN in più \emph{segmenti L2} separati (domini di broadcast distinti). 
Questo ha due effetti chiave:
\begin{enumerate}
  \item \textbf{blocca i protocolli di controllo L2} e tutto il traffico broadcast/multicast tra segmenti (ARP, STP, DHCP non attraversano il confine L3 a meno di funzioni dedicate), 
  \item rende \textbf{impossibili gli attacchi L2 tra segmenti} (eavesdropping, MAC-table attacks, VLAN hopping, MITM via ARP/DHCP) perché gli header MAC si fermano al router e l'instradamento avviene sul piano IP. 
Le stesse minacce restano possibili \emph{all'interno} di ciascun segmento se questo continua a essere uno switch L2. 
\end{enumerate}
\noindent\textit{Perché funziona.} 
Il confine L3 elimina la connettività a livello Ethernet tra i segmenti: un host in segmento A non può inviare ARP o BPDU verso segmento B, né può “floodare” unicast sconosciuti oltre il router. 
Di conseguenza, \textbf{ARP spoofing}, \textbf{MAC flooding}, manipolazioni \textbf{STP} e \textbf{VLAN hopping} non propagano tra segmenti diversi. 
Il traffico inter-segmento è visibile solo sul router (o firewall), dove si possono applicare ACL e ispezioni più robuste.

% --- Diagrammi di supporto: ARP e DHCP con confini L3 ---

\begin{lstlisting}[caption={Schema: ARP confinato dal confine L3 (il router non inoltra ARP)}]
[Host A 10.0.10.23] -- (Switch L2 VLAN10) -- [Router L3] -- (Switch L2 VLAN20) -- [Host B 10.0.20.45]

Domanda: A vuole parlare con 10.0.20.45 -> prima deve risolvere MAC.
- ARP Request "Chi ha 10.0.20.45?" \'e un BROADCAST L2 (ff:ff:ff:ff:ff:ff).
- Il BROADCAST rimane nel DOMINIO L2 (VLAN10). Il router L3 NON inoltra ARP.
- A ARP-a solo il suo gateway (10.0.10.1) per uscire dal segmento; il resto \'e routing IP.
Conclusione: ARP spoofing/flooding resta confinato entro la VLAN/segmento L2.
\end{lstlisting}

\begin{lstlisting}[caption={Schema: DHCP con relay (giaddr) oltre il confine L3}]
Client (senza IP)    Switch/VLAN10       Router L3 (DHCP Relay)             DHCP Server (172.16.0.10)

0.0.0.0:68  --[BCAST]-->  DISCOVER  --[riceve]-->  aggiunge giaddr=10.0.10.1  --[UNICAST UDP/67]-->  Server
                                                             <---[UNICAST UDP/67]---  OFFER (dest: 10.0.10.1)
Relay inoltra in VLAN10 come BROADCAST verso il client (UDP/68)
Client --[BCAST]--> REQUEST  --> Relay --[UNICAST]--> Server
                                                             <---[UNICAST]---  ACK (lease, GW, DNS)
Relay inoltra l'ACK in VLAN10 (di solito come broadcast) al client.
Esito: lease IP (es. 10.0.10.50), gateway 10.0.10.1, DNS..., registrati dal server.
Nota: senza relay, i messaggi DHCP non attraversano il confine L3.
\end{lstlisting}


\subsubsection{Controllo d’accesso (Access Control)}

\paragraph{Obiettivo.}
(i) autenticazione all’ingresso, (ii) autorizzazione fine a cosa può fare l’endpoint, (iii) enforcement sullo switch con ACL/Port Security.

\paragraph{802.1X (port-based NAC).}
\textbf{Supplicant} (host), \textbf{Authenticator} (switch) e \textbf{Authentication Server} (tipicamente RADIUS) negoziano l’accesso usando \textbf{EAP} (extensible authentication protocol) veicolato su \textbf{EAPOL} (EAP over LAN). 802.1X supporta credenziali diverse (username/password o certificati) e lega l’identità alla \emph{porta} di switch all’inizio della sessione; il controllo si appoggia a RADIUS (EAP relay/termination) e ai tipi di messaggi EAPOL (Start, EAP, Key, Logoff). In tal modo lo switch può ammettere/negare la porta e associare policy all’identità autenticata. 

\textit{Effetti attesi e limiti.} Gli switch \emph{802.1X-capable} mitigano \emph{MAC spoofing} e \emph{flooding} legando il MAC alla porta autenticata; restano però possibili attacchi fuori dal perimetro di 802.1X (es.\ \emph{ARP poisoning}) e il \emph{piggybacking} inserendo un piccolo hub/switch tra host e porta autenticata. L’autenticazione tra switch può creare un “inner core” fidato per impedire che un host si finga switch. 

\paragraph{Autorizzazione dinamica con 802.1X.}
Dopo l’\emph{auth}, il server può spingere parametri di \textbf{autorizzazione}:
\begin{itemize}
  \item \textbf{VLAN per-utente} (membership assegnata dall’identità); 
  \item \textbf{ACL per-utente} (policy di filtro associate all’identità);
  \item \textbf{UCL (User Control List)}: gruppi di utenti che condividono una stessa policy/ACL. 
\end{itemize}
Queste informazioni sono veicolate nell’esito dell’autenticazione e consumate dallo switch per configurare forwarding e filtri coerenti con l’identità. 

\paragraph{Port Security \& ACL sugli switch.}
\textbf{Port Security} limita il numero (e opzionalmente l’identità) dei MAC appresi per porta: blocca il \emph{MAC flooding} e rende più difficile l’espansione non autorizzata della LAN aggiungendo switch clandestini. Le \textbf{ACL} a livello Ethernet non sono parte dello standard, ma gli switch moderni possono filtrare su MAC sorgente/destinazione o \texttt{Ethertype}; gli switch L3 estendono il match a campi L3/L4. In laboratorio, le ACL L2 si possono esprimere con \texttt{ebtables} (NETFILTER). 

\begin{lstlisting}[language=bash,caption={Esempio minimale di binding MAC→porta con ebtables}]
ebtables -A FORWARD --in-interface eth0 -s ! a0:a0:a0:a0:a0:a0 -j DROP
ebtables -A FORWARD --in-interface eth1 -s ! b0:b0:b0:b0:b0:b0 -j DROP
ebtables -A FORWARD --in-interface eth2 -s    c0:c0:c0:c0:c0:c0 -j DROP
\end{lstlisting}
\noindent (Le ACL del laboratorio usano \texttt{ebtables}/NETFILTER; le regole si applicano per catena e per livello.)

\paragraph{Segmentazione con VLAN (complemento).}
Le \textbf{VLAN 802.1Q} riducono il dominio di broadcast e separano il traffico L2; l’efficacia dipende dalla corretta configurazione (trunk, VLAN nativa, porte \emph{access}). È pratica comune associare la VLAN all’identità 802.1X (VLAN dinamiche) e poi controllare il traffico inter-VLAN a L3. \emph{Nota: i default non sono sicuri; configurazioni errate possono abilitare VLAN hopping}.

\newpage
\maketitle
{\LARGE \textbf{Laboratorio}} \\[0.5em]

% =========================================================
\subsection{Laboratorio 2A: ACL L2 con Linux Bridge ed \texttt{ebtables} (binding MAC→porta)}

Questo laboratorio mostra come applicare un controllo d’accesso a Livello 2 vincolando,
per ogni porta, i soli indirizzi MAC ammessi (\emph{MAC→porta}). L’obiettivo è accettare
frame in ingresso solo dagli host attesi e bloccare tentativi di \emph{MAC flooding/spoofing}. 

\subsection*{Topologia di rete}
Un host Linux con tre interfacce (\texttt{eth0}, \texttt{eth1}, \texttt{eth2}) funge da \emph{bridge} L2
(“switch” emulato). Su ciascuna porta è previsto un client:
\begin{itemize}
  \item \textbf{client-1} collegato a \texttt{eth0}, MAC previsto \texttt{a0:a0:a0:a0:a0:a0};
  \item \textbf{client-2} collegato a \texttt{eth1}, MAC previsto \texttt{b0:b0:b0:b0:b0:b0};
  \item \textbf{client-3} collegato a \texttt{eth2}, MAC \texttt{c0:c0:c0:c0:c0:c0} da \emph{non} accettare.
\end{itemize}
\textbf{Goal:} accettare solo i MAC previsti su ciascuna porta; verificare che il traffico in ingresso da \texttt{client-3} sia scartato.

\subsection*{Configurazione di esempio}

\paragraph{1) Preparazione host (MAC/IP dei client).}
Esempio di impostazione (lato client) dei MAC e degli IP:
\begin{lstlisting}[language=bash,caption={Impostazione MAC/IP sui client (esempio)}]
# client-1
ip link set dev eth0 address a0:a0:a0:a0:a0:a0
ip addr add 10.0.0.1/24 dev eth0

# client-2
ip link set dev eth0 address b0:b0:b0:b0:b0:b0
ip addr add 10.0.0.2/24 dev eth0

# client-3 (non ammesso)
ip link set dev eth0 address c0:c0:c0:c0:c0:c0
ip addr add 10.0.0.3/24 dev eth0
\end{lstlisting}
(Questi valori riflettono quelli delle slide del laboratorio.)

\paragraph{2) Creazione del bridge L2 (lato “switch” Linux).} .\\
\begin{lstlisting}[language=bash,caption={Bridge L2 con tre porte}]
# crea il bridge e collega le porte
ip link add name bridge type bridge
ip link set bridge up
ip link set dev eth0 master bridge
ip link set dev eth1 master bridge
ip link set dev eth2 master bridge
\end{lstlisting}
Tutte le porte sono ora nello stesso dominio L2; senza ACL, i tre client comunicano liberamente.

\paragraph{3) ACL L2 con \texttt{ebtables} (NETFILTER).} .\\
\begin{lstlisting}[language=bash,caption={Binding MAC→porta con \texttt{ebtables}}]
# accetta su eth0 solo il MAC di client-1
ebtables -A FORWARD --in-interface eth0 -s ! a0:a0:a0:a0:a0:a0 -j DROP
# accetta su eth1 solo il MAC di client-2
ebtables -A FORWARD --in-interface eth1 -s ! b0:b0:b0:b0:b0:b0 -j DROP
# blocca su eth2 il MAC di client-3
ebtables -A FORWARD --in-interface eth2 -s    c0:c0:c0:c0:c0:c0 -j DROP

# opzionale: replica le stesse policy anche sulla chain INPUT del bridge-host
ebtables -A INPUT --in-interface eth0 -s ! a0:a0:a0:a0:a0:a0 -j DROP
ebtables -A INPUT --in-interface eth1 -s ! b0:b0:b0:b0:b0:b0 -j DROP
ebtables -A INPUT --in-interface eth2 -s    c0:c0:c0:c0:c0:c0 -j DROP
\end{lstlisting}
Le ACL sono espresse nel framework \textbf{NETFILTER}: \texttt{ebtables} opera a L2 (MAC/EtherType),
\texttt{iptables}/\texttt{ip6tables} a L3/L4.

\subsection*{Funzionamento del laboratorio}
Le regole applicano un vincolo di identità L2: ogni porta accetta solo i frame dal MAC atteso;
frame da MAC diversi vengono droppati (\emph{anti-spoofing} di base, freno al \emph{flooding}).

\subsection*{Verifica del comportamento}
\begin{itemize}
  \item \textbf{client-1} $\leftrightarrow$ \textbf{client-2}: ping riuscito.
  \item \textbf{client-3} $\to$ altri: i pacchetti in ingresso sono scartati (nessuna risposta al ping).
\end{itemize}
Suggerimento: osserva i contatori \texttt{ebtables -L --Lc} mentre invii traffico di test.

\subsection*{Osservazione pratica}
Questo schema è didattico ma realistico: i vendor implementano meccanismi simili (\emph{Port Security}, ACL L2)
sugli switch per contenere spoofing/flooding alla porta. 

% =========================================================
\subsection{Laboratorio 2B: MACsec su Linux (confidenzialità, integrità e anti-replay a L2)}

Questo laboratorio mostra come attivare \textbf{MACsec} (IEEE~802.1AE) tra due host Linux
per proteggere il traffico L2 con integrità, opzionalmente cifratura e protezione \emph{anti-replay}.

\subsection*{Topologia di rete}
Due host (\textbf{client1}, \textbf{client2}) collegati alla stessa LAN Ethernet. Ogni host crea un’interfaccia
\texttt{macsec0} legata alla NIC fisica (\texttt{eth0}) e configura \emph{Secure Channel} (SC) e
\emph{Security Association} (SA) con chiavi simmetriche.

\subsection*{Configurazione di esempio}

\paragraph{Host A (\textbf{client1}).} .\\
\begin{lstlisting}[language=bash,caption={MACsec lato client1}]
ip link add link eth0 macsec0 type macsec
# SA TX di client1 (chiave K1), SA RX verso il MAC di client2 (chiave K2)
ip macsec add macsec0 tx sa 0 pn 1 on key 01 09876543210987654321098765432109
ip macsec add macsec0 rx address b0:b0:b0:b0:b0:b0 port 1
ip macsec add macsec0 rx address b0:b0:b0:b0:b0:b0 port 1 sa 0 pn 1 on key 02 12345678901234567890123456789012
ip link set macsec0 up
ip addr add 10.100.0.1/24 dev macsec0
\end{lstlisting}

\paragraph{Host B (\textbf{client2}).} .\\
\begin{lstlisting}[language=bash,caption={MACsec lato client2}]
ip link add link eth0 macsec0 type macsec
# SA TX di client2 (chiave K2), SA RX verso il MAC di client1 (chiave K1)
ip macsec add macsec0 tx sa 0 pn 1 on key 02 12345678901234567890123456789012
ip macsec add macsec0 rx address a0:a0:a0:a0:a0:a0 port 1
ip macsec add macsec0 rx address a0:a0:a0:a0:a0:a0 port 1 sa 0 pn 1 on key 01 09876543210987654321098765432109
ip link set macsec0 up
ip addr add 10.100.0.2/24 dev macsec0
\end{lstlisting}

\paragraph{Opzioni di sicurezza e test.} .\\
\begin{lstlisting}[language=bash,caption={Cifratura e anti-replay; test con ping}]
# Cifratura dei frame e protezione anti-replay (opzionali)
ip link set macsec0 type macsec encrypt on
ip link set macsec0 type macsec replay on

# Test
ping 10.100.0.2   # da client1 (verifica su Wireshark l'intestazione MACsec)
\end{lstlisting}
Con questa configurazione, MACsec fornisce integrità; attivando \texttt{encrypt on} aggiungi confidenzialità.
La suite predefinita è \emph{GCM-AES-128}; \emph{GCM-AES-256} è disponibile in estensione. La gestione chiavi
dinamica è demandata a 802.1X-2010 (fuori dallo scopo di questo lab). 

\subsection*{Funzionamento del laboratorio}
MACsec inserisce, tra MAC header e payload, un \emph{Security Tag} e calcola un ICV (codice d’integrità);
i frame risultano protetti contro manomissioni e, se abilitata, cifrati in transito. 

\subsection*{Verifica del comportamento}
\begin{itemize}
  \item \textbf{Integrità only}: con \texttt{encrypt off} vedi i Layer-3 in chiaro ma i frame hanno tag/ICV MACsec.
  \item \textbf{Cifratura on}: abilita \texttt{encrypt on} e osserva in Wireshark che il payload non è leggibile.
  \item \textbf{Anti-replay}: abilita \texttt{replay on} e verifica che ritrasmissioni artificiose vengano scartate.
\end{itemize}
Suggerimento: cattura sul link fisico \texttt{eth0} per osservare l’incapsulamento MACsec.

\subsection*{Osservazione pratica}
MACsec elimina \emph{sniffing/MITM} sul filo (anche in caso di flooding per timeout CAM), ma non impedisce DoS
né abusi da host già autorizzati: serve comunque \emph{hardening} (ACL/DAI/Port Security) e monitoraggio. 


\subsubsection{Secure Address Resolution (IPv4/IPv6)}

\paragraph{IPv4: dal piano di switching ai vincoli per-presa.}
\textbf{DHCP Snooping} costruisce una tabella \texttt{(IP, MAC, porta, VLAN)} osservando i messaggi DHCP leciti; questa base di conoscenza alimenta la \textbf{Dynamic ARP Inspection (DAI)} che blocca risposte ARP incoerenti con i binding noti. \textbf{IP Source Guard} applica il binding direttamente a L2, filtrando i frame con IP sorgente non atteso su quella porta. \emph{Limite pratico:} lo \emph{scope} è per-switch; un apparato vede solo i lease che attraversano le sue porte. Progettare i domini DHCP in modo che il traffico passi dove serve lo snooping.

\paragraph{IPv4: proposte crittografiche.}
\textbf{S-ARP} estende ARP con un campo di autenticazione e un’infrastruttura di gestione chiavi; in alternativa, si può “estendere” la protezione L2 fornita da \textbf{MACsec} fino all’endpoint e al multicast. Queste soluzioni sono robuste sul piano tecnico, ma più complesse da distribuire. 

\paragraph{IPv6: NDP sicuro.}
In IPv6, la sicurezza della risoluzione passa da \textbf{SEND} (\emph{Secure Neighbor Discovery}, RFC~3971/6494): usa \textbf{CGA} (Cryptographically Generated Addresses) e opzioni firmate per autenticare i messaggi NDP, fornendo un meccanismo alternativo a IPsec, specifico per la discovery. Nella pratica operativa, resta poco adottato; in molti contesti si preferiscono controlli sullo switch (es.\ RA/DHCPv6 Guard) per mitigare \emph{router/neighbor spoofing}.

\subsubsection{Security Monitoring}

\paragraph{Firewall \& DPI.}
I firewall delimitano il traffico tra segmenti (casi avanzati di ACL) e, nei prodotti moderni, possono operare a più livelli con \textbf{Deep Packet Inspection} e ricostruzione di sessione applicativa. Nel dominio L2, il concetto di “Ethernet firewall” è assorbito da \emph{ACL di switch} e dallo stack \emph{NETFILTER} (ebtables/iptables); i firewall tradizionali gestiscono i livelli superiori.

\paragraph{IDS/IPS e accesso al traffico.}
\textbf{IDS/IPS} identificano attacchi tramite firme/anomalie e necessitano di vedere i pacchetti: o in \emph{inline} (come un firewall) oppure tramite \textbf{port mirroring} (\emph{SPAN}), che copia il traffico di porte selezionate verso una porta di ascolto. I tap passivi su rame/fibra sono alternative poco rilevabili, ma richiedono accesso fisico.

\paragraph{Dove il monitoring aiuta davvero.}
Il mirroring/IDS rende \emph{osservabili} attacchi tipici L2 (es.\ \emph{MAC flooding}, \emph{double–tagging}): molte firme sono facili da catturare; l’analisi DPI può correlare eventi di \emph{poisoning} e \emph{replay}. In parallelo, le \emph{ACL} sugli switch limitano la superficie a L2.

\paragraph{Messaggio chiave (dalle slide).}
Il livello di sicurezza cresce con l’impegno amministrativo: una Ethernet \emph{out–of–the–box} resta insicura (MAC flooding, ARP spoofing, STP attacks sono banali). Anche con ACL/802.1X, l’ARP spoofing resta possibile senza \emph{DHCP Snooping + ARP Inspection}; MACsec risolve \emph{sniffing/MiTM} ma non \emph{DoS}/analisi del traffico. \emph{Conclusione:} servono difese stratificate e configurazioni accurate.
