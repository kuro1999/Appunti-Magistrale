\newpage
\section{NET\_07 - Secure Protocols and Overlay VPNs}

Questa sezione analizza i protocolli crittografici utilizzati per proteggere le comunicazioni di rete,
con particolare attenzione alle \textbf{Overlay VPN}, al funzionamento di \textbf{OpenVPN} e
all’interazione con \textbf{NETFILTER}. Si esaminano inoltre le tecniche crittografiche applicate ai
diversi livelli dello stack ISO/OSI e il modo in cui esse compensano l’assenza di sicurezza nativa nei
protocolli Internet.

\subsection{Recap: Protocolli di Sicurezza di Rete}

I protocolli di rete come IP, TCP e UDP sono intrinsecamente insicuri. Al fine di garantire
confidenzialità, integrità e autenticazione, vengono introdotti protocolli crittografici ai diversi livelli
dello stack:

\begin{itemize}
    \item \textbf{Livello applicativo}: SSH per accesso remoto sicuro.
    \item \textbf{Livello di trasporto}: TLS per protezione end-to-end applicativa.
    \item \textbf{Livello di rete}: IPSec per proteggere comunicazioni host-to-host o gateway-to-gateway.
\end{itemize}

Essi forniscono:
\begin{itemize}
    \item \textit{Confidenzialità}: cifratura del traffico.
    \item \textit{Integrità}: MAC e HMAC.
    \item \textit{Autenticazione}: scambio di chiavi e certificati.
\end{itemize}

\subsection{Overlay VPN}

Una \textbf{Overlay VPN} permette di costruire una rete privata virtuale sopra una rete pubblica
non sicura (tipicamente Internet). La topologia della VPN è ``virtuale'' e indipendente dalla
topologia fisica sottostante.

Gli obiettivi principali sono:

\begin{itemize}
    \item creare un canale privato logico fra endpoint remoti;
    \item garantire integrità, autenticità e confidenzialità del traffico;
    \item nascondere ai router intermedi la struttura della rete interna.
\end{itemize}

\subsection{Topologia Overlay e Sottostante nelle VPN}

La figura seguente illustra la differenza tra la
\textbf{topologia overlay} di una VPN e la \textbf{topologia sottostante} (underlay) fornita
da Internet.

\begin{figure}[H]
    \centering
    \includegraphics[width=0.75\linewidth]{immagini/NET_07/overlay}
    \caption{Differenza tra topologia Overlay VPN e Underlay fisica.}
    \label{fig:overlay}
\end{figure}

Nella parte superiore della figura è rappresentata la \textbf{Overlay Topology}:
i nodi della VPN (Node~1, Node~2, Node~3) sono connessi fra loro tramite indirizzi
virtuali nella rete privata \texttt{10.0.0.0/24}. Dal punto di vista logico:

\begin{itemize}
    \item Node~1 ha indirizzo VPN \texttt{10.0.0.1/24};
    \item Node~2 ha indirizzo VPN \texttt{10.0.0.2/24};
    \item Node~3 ha indirizzo VPN \texttt{10.0.0.3/24}.
\end{itemize}

La VPN realizza quindi una mesh logica in cui ciascun nodo è raggiungibile tramite il tunnel,
indipendentemente dall'effettiva posizione fisica nella rete reale.

\subsubsection{Underlay: topologia reale}

La parte inferiore della figura mostra la \textbf{topologia sottostante}, ovvero Internet.
Ogni nodo si trova in una rete IP completamente diversa:

\begin{itemize}
    \item Node~1 è fisicamente nel segmento \texttt{192.168.0.3/24};
    \item Node~2 si trova nella rete pubblica \texttt{1.2.3.4/22};
    \item Node~3 appartiene al blocco \texttt{6.7.8.9/16}.
\end{itemize}

Sebbene tali indirizzi non siano correlati fra loro, ognuno di essi è raggiungibile tramite Internet e
può instaurare un tunnel verso gli altri nodi della VPN.

\subsubsection{Relazione Overlay–Underlay}

L'Overlay poggia sul servizio di connettività IP fornito dall'Underlay:

\begin{itemize}
    \item Il routing nella VPN usa solo gli indirizzi virtuali \texttt{10.0.0.x}.
    \item Il routing dell’underlay usa gli indirizzi IP reali pubblici/privati.
    \item I router dell’underlay non conoscono la VPN né la sua topologia interna.
\end{itemize}

Ogni pacchetto inviato da Node~1 a Node~3 avviene secondo i seguenti passaggi:

\begin{enumerate}
    \item \textbf{Node~1 costruisce un pacchetto con destinazione 10.0.0.3}.
          Questo pacchetto appartiene alla topologia overlay.
    \item \textbf{Il software VPN lo incapsula} in un pacchetto IP esterno con destinazione
          l'indirizzo fisico dell'altro nodo (es.\ \texttt{6.7.8.9}).
    \item \textbf{Internet effettua il forwarding} basandosi esclusivamente sull'header esterno.
    \item \textbf{Node~3 decapsula} il pacchetto restituendo l’header VPN interno.
\end{enumerate}

Questo meccanismo permette ai nodi di comunicare come se fossero nella stessa rete locale privata,
pur non condividendo alcuna caratteristica della rete fisica sottostante.

\subsubsection{Vantaggi dell’Overlay}

La separazione Overlay/Underlay offre:

\begin{itemize}
    \item \textbf{Indipendenza dalla topologia fisica}: la VPN può unire nodi sparsi nel mondo.
    \item \textbf{Indirizzamento uniforme}: la rete privata \texttt{10.0.0.0/24} è indipendente dagli IP reali.
    \item \textbf{Sicurezza}: l'incapsulamento nel tunnel permette cifratura e autenticazione.
    \item \textbf{Gestione semplificata}: il routing interno è controllato interamente dal software VPN.
\end{itemize}



\subsubsection{Problema 1: Creazione della Rete Privata}

Per collegare nodi remoti come se fossero nella stessa LAN, è necessario l'uso del \textbf{tunneling}.
Il tunneling consiste nell'incapsulare un pacchetto IP interno come payload di un pacchetto IP esterno.

I router della rete pubblica vedono solo l’header esterno e fanno forwarding ignorando la struttura
interna della VPN.


\begin{figure}[H]
    \centering
    \includegraphics[width=.85\linewidth]{immagini/NET_07/tunnel}
    \caption{Esempio di incapsulamento e decapsulamento in una Overlay VPN: un pacchetto con
    destinazione privata \texttt{10.0.0.2} viene incapsulato dal gateway del Site~1 in un pacchetto
    esterno con destinazione pubblica \texttt{2.2.2.2}, attraversa Internet e viene decapsulato dal
    gateway del Site~2, che riconsegna il pacchetto interno all'host VPN remoto.}
    \label{fig:tunnel}
\end{figure}

\subsubsection{Incapsulamento e Decapsulamento dei Pacchetti nella Overlay VPN}

La figura mostra il funzionamento del meccanismo di \textbf{tunneling} usato da una Overlay VPN.
Due siti remoti, Site~1 e Site~2, appartengono alla stessa rete privata virtuale \texttt{10.0.0.0/24}
pur essendo connessi tramite Internet.

Il traffico generato dagli host della VPN utilizza indirizzi privati non instradabili pubblicamente
(ad esempio \texttt{10.0.0.1} → \texttt{10.0.0.2}). Poiché tali indirizzi non possono essere inoltrati
nativamente attraverso la rete pubblica, i gateway VPN applicano un meccanismo di incapsulamento:

\begin{enumerate}
    \item L'host \texttt{10.0.0.1} invia un pacchetto destinato a \texttt{10.0.0.2}.
    \item Il gateway del Site~1 incapsula il pacchetto interno in un nuovo pacchetto IP esterno:
    \begin{itemize}
        \item \textbf{IP interno}: sorgente \texttt{10.0.0.1}, destinazione \texttt{10.0.0.2};
        \item \textbf{IP esterno}: sorgente \texttt{1.1.1.1}, destinazione \texttt{2.2.2.2}.
    \end{itemize}
    \item Il pacchetto esterno attraversa Internet: i router dell’underlay vedono solo l’header esterno
          e inoltrano il traffico verso \texttt{2.2.2.2}.
    \item Il gateway del Site~2, una volta ricevuto il pacchetto, effettua il \textbf{decapsulamento}
          rimuovendo l’header esterno.
    \item Il pacchetto interno viene reinserito nella rete privata del Site~2 e consegnato all’host
          \texttt{10.0.0.2}.
\end{enumerate}

Questo processo permette ai due siti di comportarsi come se fossero connessi alla stessa LAN privata,
garantendo trasparenza di indirizzamento e sicurezza del trasporto grazie alla cifratura del tunnel.



\subsubsection{Problema 2: Sicurezza del Tunnel}

Per proteggere il contenuto del pacchetto incapsulato, è necessario applicare protocolli crittografici.
Una Overlay VPN combina quindi:

\begin{itemize}
    \item \textit{incapsulamento} (tunneling)
    \item \textit{crittografia} (TLS, IPSec, ecc.)
    \item \textit{autenticazione} (certificati)
\end{itemize}


\begin{figure}[H]
    \centering
    \includegraphics[width=.9\linewidth]{immagini/NET_07/secu}
    \caption{Confronto tra cifratura \textit{Gateway-to-Gateway} e \textit{End-to-End}.
    Nel modello GW-to-GW il traffico viene cifrato e decifrato unicamente dai gateway VPN,
    mentre gli host interni comunicano in chiaro nella propria LAN; nel modello End-to-End
    la cifratura è applicata direttamente dagli host, garantendo protezione completa del
    contenuto lungo l’intero percorso.}
    \label{fig:secu}
\end{figure}


\subsubsection{Approfondimento sui dispositivi TUN e TAP}

OpenVPN interagisce con il sistema operativo attraverso dispositivi di rete virtuali.
Questi dispositivi non corrispondono a interfacce fisiche, ma sono creati in software
per permettere al processo OpenVPN di inviare e ricevere pacchetti come se fosse una
vera scheda di rete. Esistono due tipologie principali:

\paragraph{Dispositivo TUN}
Un'interfaccia \textbf{TUN} opera a \textit{livello 3} (IP) del modello OSI.
Caratteristiche principali:

\begin{itemize}
    \item gestisce solo pacchetti IP (IPv4/IPv6), non frame Ethernet;
    \item non supporta ARP, broadcast o multicast Ethernet;
    \item crea un tunnel \textit{point-to-point} tra due endpoint (tipico per VPN site-to-site e client--server);
    \item l'applicazione (OpenVPN) riceve direttamente il pacchetto IP da incapsulare nel tunnel.
\end{itemize}

Un dispositivo \textbf{TUN} è appropriato quando si vuole creare un tunnel che trasporta
esclusivamente traffico IP. Poiché opera a livello 3, gestisce solo pacchetti IPv4/IPv6 e non
propaga frame Ethernet, ARP o traffico broadcast. Questo lo rende ideale per VPN point-to-point
o site-to-site in cui è sufficiente instradare traffico IP tra reti distinte senza ricreare una LAN
completa.
Esempio: un’azienda collega la sede centrale (\texttt{10.0.0.0/24}) con una filiale remota
(\texttt{10.0.1.0/24}) tramite un tunnel TUN. Il tunnel trasporta solo pacchetti IP instradati
tra le due reti, senza estendere la LAN e senza gestire traffico broadcast o protocolli di
livello 2.




%%%%%%%%    TAP %%%%%%%%%%%
\paragraph{Dispositivo TAP}
Un'interfaccia \textbf{TAP} opera a \textit{livello 2} (Ethernet).
Caratteristiche principali:

\begin{itemize}
    \item gestisce frame Ethernet completi (header L2 + payload);
    \item supporta ARP, broadcast e multicast;
    \item consente di estendere una LAN reale a distanza, creando una ``LAN virtuale'';
    \item consente di far circolare protocolli non-IP (ad esempio NetBIOS).
\end{itemize}


Un dispositivo \textbf{TAP}, operando a livello 2, trasporta interi frame Ethernet e riproduce il
comportamento di una vera interfaccia di rete. Supporta ARP, broadcast e multicast e permette
di estendere un dominio Ethernet tra due siti remoti. È quindi indicato quando la VPN deve
comportarsi come un’unica LAN, oppure quando occorre trasportare protocolli non-IP o servizi
che dipendono dal livello 2.

Esempio: due uffici remoti devono condividere la stessa LAN Ethernet per permettere
l’erogazione di DHCP e il funzionamento di protocolli basati su broadcast, come ARP o
NetBIOS. Utilizzando TAP, l’intera rete di livello 2 viene estesa attraverso il tunnel,
come se gli host fossero collegati allo stesso switch.

\begin{figure}[H]
    \centering
    \includegraphics[width=.8\linewidth]{immagini/NET_07/openvpn}
    \caption{Funzionamento di un client OpenVPN con interfaccia virtuale TUN.
    Il traffico destinato alla rete privata virtuale \texttt{10.0.0.0/24} viene instradato
    verso l'interfaccia \texttt{tun0}, incapsulato dal processo OpenVPN e successivamente
    inviato attraverso l’interfaccia fisica \texttt{eth0} verso il server VPN, situato
    all’indirizzo pubblico \texttt{2.2.2.2}.}
    \label{fig:openvpn}
\end{figure}

\subsubsection{Flusso di instradamento e incapsulamento con interfacce TUN}

La figura ~\ref{fig:openvpn} mostra il comportamento di un host configurato come client OpenVPN quando utilizza
un'interfaccia virtuale \texttt{tun0}. L’obiettivo è instradare traffico IP privato destinato alla rete
overlay \texttt{10.0.0.0/24} verso il server VPN remoto.

\paragraph{Configurazione dell’host}

Il client possiede:

\begin{itemize}
    \item un’interfaccia virtuale \texttt{tun0} con indirizzo \texttt{10.0.0.1/30};
    \item un’interfaccia fisica \texttt{eth0} configurata in \texttt{192.168.0.100/24};
    \item una tabella di routing che associa l’intera rete VPN \texttt{10.0.0.0/24} al next hop
          \texttt{10.0.0.2} sull'interfaccia \texttt{tun0}.
\end{itemize}

Questa configurazione è generata automaticamente dal processo OpenVPN al momento della
connessione con il server.

\paragraph{Scelta dell’interfaccia di uscita}

Quando un’applicazione locale invia un pacchetto verso un indirizzo VPN, ad esempio
\texttt{10.0.0.2}, la tabella di routing determina che:

\begin{itemize}
    \item il pacchetto deve uscire da \texttt{tun0};
    \item l’indirizzo sorgente sarà \texttt{10.0.0.1};
    \item il next hop del sottotunnel è \texttt{10.0.0.2}.
\end{itemize}

Poiché \texttt{tun0} è una point-to-point interface, ogni pacchetto IP consegnato al kernel
viene immediatamente intercettato dal processo OpenVPN.

\paragraph{Incapsulamento e invio verso il server VPN}

Il processo OpenVPN esegue i seguenti passi:

\begin{enumerate}
    \item legge dal dispositivo \texttt{/dev/tun0} il pacchetto IP destinato a \texttt{10.0.0.2};
    \item incapsula tale pacchetto in un datagramma UDP o TCP, secondo la configurazione;
    \item costruisce un nuovo header IP esterno con destinazione \texttt{2.2.2.2},
          indirizzo pubblico del server VPN;
    \item invia il pacchetto esterno attraverso l'interfaccia fisica \texttt{eth0}, seguendo
          il default gateway locale \texttt{192.168.0.1}.
\end{enumerate}

Durante questo processo i router della rete sottostante (Internet) vedono soltanto il pacchetto
esterno, mentre il pacchetto interno rimane protetto dal tunnel.

\paragraph{Effetto logico}

Dal punto di vista delle applicazioni locali, la presenza dell’interfaccia TUN simula una
connessione diretta alla rete privata \texttt{10.0.0.0/24}.
La gestione delle chiavi, la cifratura, l’incapsulamento e l’inoltro sono completamente
trasparenti all’utente e vengono gestiti dal processo OpenVPN in user space.

\begin{figure}[H]
    \centering
    \includegraphics[width=.8\linewidth]{immagini/NET_07/openvpn2}
    \caption{OpenVPN host to server with TUN virtual interfaces}
\end{figure}



\subsubsection{Handshake e Modello di Sicurezza}

OpenVPN (in \emph{TLS mode}) utilizza \textbf{TLS} per proteggere il \textit{control channel},
ossia il canale logico su cui viaggiano le informazioni di configurazione e il materiale
cifrante.:contentReference[oaicite:0]{index=0}

Il primo passo per costruire una VPN con OpenVPN è l’istituzione di una \textbf{PKI}
(\textit{Public Key Infrastructure}) composta da:​:contentReference[oaicite:1]{index=1}
\begin{itemize}
    \item una coppia chiave privata + certificato X.509 per il server;
    \item una coppia chiave privata + certificato X.509 per ciascun client;
    \item una \textbf{Certification Authority} (CA) con propria chiave e certificato,
          usata per firmare tutti i certificati server e client.
\end{itemize}

L’autenticazione è \textbf{mutua}:
\begin{itemize}
    \item il client verifica che il certificato presentato dal server sia stato firmato dalla CA
          e che i campi del certificato (es.\ Common Name, tipo \emph{server}) siano coerenti;
    \item il server verifica in modo analogo il certificato del client (tipo \emph{client}).:contentReference[oaicite:2]{index=2}
\end{itemize}

Una volta completato il TLS handshake:
\begin{enumerate}
    \item viene stabilita una sessione TLS autenticata sul \emph{control channel};
    \item tramite TLS vengono negoziate e scambiate le \textbf{chiavi di sessione}
          (materiale cifrante) per il \textbf{data channel};
    \item il \textit{data channel} utilizza cifratura simmetrica (es.\ AES-GCM) e HMAC
          per garantire confidenzialità, integrità e protezione dal replay.:contentReference[oaicite:3]{index=3}
\end{enumerate}

\paragraph{OpenVPN security model}

Il modello di sicurezza basato su CA presenta alcune proprietà rilevanti:
\begin{itemize}
    \item \textbf{Scalabilità}: il server necessita solo del \textit{proprio} certificato/chiave
          e del certificato della CA; non deve conoscere a priori tutti i certificati dei client.
          Accetta solo client le cui credenziali sono firmate dalla CA.
    \item \textbf{Isolamento della CA}: la chiave privata della CA (la più sensibile)
          può risiedere su una macchina separata e anche offline; il server usa solo il
          certificato della CA per verificare le firme, non la chiave privata.
    \item \textbf{Revoca granulare}: se la chiave di un client viene compromessa, è sufficiente
          inserire il relativo certificato nella \textbf{CRL} (Certificate Revocation List).
          OpenVPN può consultare la CRL per rifiutare selettivamente quei certificati senza
          rigenerare l’intera PKI.
    \item \textbf{Autorizzazione basata sul certificato}: il server può applicare politiche
          di accesso specifiche in base ai campi del certificato (es.\ \emph{Common Name}),
          associando a ciascun client determinate reti raggiungibili o privilegi.
\end{itemize}

Dal punto di vista del trasporto, OpenVPN può usare sia \textbf{UDP} che \textbf{TCP}; in
modalità UDP implementa \textbf{TLS over UDP}, ed è la scelta raccomandata quando possibile,
per evitare il problema del “TCP over TCP”.:contentReference[oaicite:5]{index=5}

\subsubsection{Routing nella VPN}

La topologia logica tipica è \textbf{Hub-and-Spoke}: il server agisce da nodo centrale
(\emph{hub}), mentre i client sono gli \emph{spoke} che stabiliscono un tunnel verso il server.
La comunicazione client–client avviene passando attraverso il server.:contentReference[oaicite:6]{index=6}

OpenVPN utilizza tre direttive fondamentali, che agiscono su piani di routing diversi:​:contentReference[oaicite:7]{index=7}
\begin{itemize}
    \item \verb|push "route net_addr net_mask"|: il server invia ai client, durante
          l’handshake, una rotta da installare nella \textbf{tabella di routing reale}
          del client (underlay). Ogni client aggiunge un entry del tipo:
          \texttt{net\_addr/net\_mask via <peer\_tun> dev tun0}.
    \item \verb|route net_addr net_mask|: installa una rotta nella tabella di routing del
          \textbf{server} (underlay), indicando che una certa rete deve essere inoltrata
          verso l’interfaccia TUN/TAP gestita da OpenVPN.
    \item \verb|iroute net_addr net_mask|: viene usata nei file di \verb|client-config-dir|
          specifici per client (uno per ogni CN) e influenza l’\textbf{overlay routing}
          gestito dal processo OpenVPN. Serve a dire al server quale rete overlay è
          raggiungibile tramite quale tunnel (es.\ \texttt{10.0.0.0/24} via \emph{client2}).
\end{itemize}

Il server decide come smistare i pacchetti tra i vari client combinando:
\begin{itemize}
    \item le rotte underlay (\verb|route|, \verb|push route|) installate nel kernel;
    \item la tabella di routing overlay interna, popolata tramite \verb|iroute| e i file in
          \verb|client-config-dir|.
\end{itemize}


%%%%%%%%%LABORATORIO
% Da qui in poi iniziano esplicitamente le slide di laboratorio: "Lab 8: Overlay VPN with OpenVPN"
\newpage
\maketitle
{\LARGE \textbf{Laboratorio}} \\[0.5em]
\subsection{Lab 8: Overlay VPN con OpenVPN}

In questo laboratorio si realizza una VPN overlay utilizzando OpenVPN in modalità TLS.
L'obiettivo è costruire una topologia \textit{hub-and-spoke} in cui il server VPN funge da nodo
centrale e i client accedono a reti remote instradate attraverso il tunnel.

\subsubsection{Topologia}

La figura~\ref{fig:vpn_topology} mostra la topologia logica (overlay) e quella fisica
(underlay). La VPN utilizza lo spazio di indirizzamento virtuale
\texttt{192.168.100.0/24}, assegnando:

\begin{itemize}
    \item \textbf{server}: \texttt{192.168.100.1};
    \item \textbf{client1}: \texttt{192.168.100.101};
    \item \textbf{client2}: \texttt{192.168.100.105}.
\end{itemize}

\begin{figure}[h]
    \centering
    \includegraphics[width=\linewidth]{immagini/NET_07/vpn_topology}
    \caption{Topologia del laboratorio: overlay VPN OpenVPN
    (\texttt{192.168.100.0/24}) e relativa topologia fisica.}
    \label{fig:vpn_topology}
\end{figure}

L’overlay VPN permette a client e server di comunicare direttamente attraverso il tunnel,
mentre l’underlay è costituito da più domini di rete collegati da router R1 e R2.

\subsubsection{Obiettivo}

Il laboratorio ha i seguenti obiettivi:

\begin{itemize}
    \item permettere la comunicazione sicura client--server attraverso la VPN;
    \item abilitare la comunicazione client--client tramite il server (hub-and-spoke);
    \item rendere raggiungibili, tramite la VPN:
          \begin{itemize}
              \item la rete dietro il server (\texttt{192.168.0.0/24});
              \item la rete dietro client2 (\texttt{10.0.0.0/24}).
          \end{itemize}
    \item verificare l'instradamento overlay mediante ping e analisi delle routing table.
\end{itemize}

\subsubsection{Configurazione di rete (underlay)}

Ogni host del laboratorio presenta una configurazione IP coerente con la topologia fisica.
Gli elementi principali:

\begin{itemize}
    \item \textbf{server}: interfaccia fisica in \texttt{1.0.0.0/24}, default gateway R2;
    \item \textbf{client1}: rete \texttt{192.168.1.0/24}, collegato a R1;
    \item \textbf{client2}: rete \texttt{10.0.0.0/24}, collegato a R2;
    \item \textbf{R1 e R2}: connessi tramite link \texttt{1.2.0.0/30} e \texttt{2.2.0.0/30}.
\end{itemize}

Tutti gli host devono avere connettività IP verso il server VPN (\texttt{1.0.0.2}).

\subsubsection{Configurazione OpenVPN}

OpenVPN utilizza tre categorie di direttive fondamentali:

\begin{itemize}
    \item \verb|server ...| e \verb|ifconfig-push| per l’assegnazione degli indirizzi VPN;
    \item \verb|push "route ..."| per distribuire le rotte overlay ai client;
    \item \verb|iroute ...| nei file di \texttt{ccd/} per indicare al server quali reti
          sono raggiungibili tramite ciascun client.
\end{itemize}

\paragraph{Configurazione del server}

\begin{verbatim}
server 192.168.100.0 255.255.255.0
push "route 192.168.0.0 255.255.255.0"
push "route 10.0.0.0 255.255.255.0"
route 10.0.0.0 255.255.255.0
client-config-dir ccd
client-to-client
\end{verbatim}

\paragraph{File ccd/client1}

\begin{verbatim}
ifconfig-push 192.168.100.101 192.168.100.102
\end{verbatim}

\paragraph{File ccd/client2}

\begin{verbatim}
ifconfig-push 192.168.100.105 192.168.100.106
iroute 10.0.0.0 255.255.255.0
\end{verbatim}

In questo modo il server sa che la rete \texttt{10.0.0.0/24} è raggiungibile tramite il tunnel
stabilito con \texttt{client2}.

\subsubsection{Test di connettività}

Per verificare il corretto funzionamento del laboratorio vengono eseguiti i seguenti test:

\paragraph{1) client1 → server}

Ping verso \texttt{192.168.100.1}.
Il traffico attraversa:

\begin{enumerate}
    \item la routing table locale (destinazione rete VPN → \texttt{tun0});
    \item il tunnel UDP verso il server;
    \item decapsulamento e reiniezione nel network stack del server.
\end{enumerate}

\paragraph{2) client1 → client2}

Il traffico passa dal server (hub), che inoltra verso il client appropriato tramite
\verb|client-to-client|.

\paragraph{3) client1 → client-A (dietro client2)}

L’overlay routing utilizza:

\begin{itemize}
    \item \verb|push route 10.0.0.0/24| per far conoscere la rete ai client;
    \item \verb|iroute 10.0.0.0/24| per dire al server che la rete è dietro client2.
\end{itemize}

Il tunnel viene utilizzato per attraversare l’underlay fino a raggiungere la rete remota.
